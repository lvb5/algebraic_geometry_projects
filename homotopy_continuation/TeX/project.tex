\documentclass[11pt]{article}

\usepackage{newunicodechar}
    \usepackage{tikzsymbols}  % provides \Snowman
    \newunicodechar{λ}{$\lambda$}

    \usepackage[breakable]{tcolorbox}
    \usepackage{parskip} % Stop auto-indenting (to mimic markdown behaviour)
    
    \usepackage{iftex}
    \ifPDFTeX
    	\usepackage[T1]{fontenc}
    	\usepackage{mathpazo}
    \else
    	\usepackage{fontspec}
    \fi
    

    % Basic figure setup, for now with no caption control since it's done
    % automatically by Pandoc (which extracts ![](path) syntax from Markdown).
    \usepackage{graphicx}
    % Maintain compatibility with old templates. Remove in nbconvert 6.0
    \let\Oldincludegraphics\includegraphics
    % Ensure that by default, figures have no caption (until we provide a
    % proper Figure object with a Caption API and a way to capture that
    % in the conversion process - todo).
    \usepackage{caption}
    \DeclareCaptionFormat{nocaption}{}
    \captionsetup{format=nocaption,aboveskip=0pt,belowskip=0pt}

    \usepackage{float}
    \floatplacement{figure}{H} % forces figures to be placed at the correct location
    \usepackage{xcolor} % Allow colors to be defined
    \usepackage{enumerate} % Needed for markdown enumerations to work
    \usepackage{geometry} % Used to adjust the document margins
    \usepackage{amsmath} % Equations
    \usepackage{amssymb} % Equations
    \usepackage{textcomp} % defines textquotesingle
    % Hack from http://tex.stackexchange.com/a/47451/13684:
    \AtBeginDocument{%
        \def\PYZsq{\textquotesingle}% Upright quotes in Pygmentized code
    }
    \usepackage{upquote} % Upright quotes for verbatim code
    \usepackage{eurosym} % defines \euro
    \usepackage[mathletters]{ucs} % Extended unicode (utf-8) support
    \usepackage{fancyvrb} % verbatim replacement that allows latex
    \usepackage{grffile} % extends the file name processing of package graphics 
                         % to support a larger range
    \makeatletter % fix for old versions of grffile with XeLaTeX
    \@ifpackagelater{grffile}{2019/11/01}
    {
      % Do nothing on new versions
    }
    {
      \def\Gread@@xetex#1{%
        \IfFileExists{"\Gin@base".bb}%
        {\Gread@eps{\Gin@base.bb}}%
        {\Gread@@xetex@aux#1}%
      }
    }
    \makeatother
    \usepackage[Export]{adjustbox} % Used to constrain images to a maximum size
    \adjustboxset{max size={0.9\linewidth}{0.9\paperheight}}

    % The hyperref package gives us a pdf with properly built
    % internal navigation ('pdf bookmarks' for the table of contents,
    % internal cross-reference links, web links for URLs, etc.)
    \usepackage{hyperref}
    % The default LaTeX title has an obnoxious amount of whitespace. By default,
    % titling removes some of it. It also provides customization options.
    \usepackage{titling}
    \usepackage{longtable} % longtable support required by pandoc >1.10
    \usepackage{booktabs}  % table support for pandoc > 1.12.2
    \usepackage[inline]{enumitem} % IRkernel/repr support (it uses the enumerate* environment)
    \usepackage[normalem]{ulem} % ulem is needed to support strikethroughs (\sout)
                                % normalem makes italics be italics, not underlines
    \usepackage{mathrsfs}
    

    
    % Colors for the hyperref package
    \definecolor{urlcolor}{rgb}{0,.145,.698}
    \definecolor{linkcolor}{rgb}{.71,0.21,0.01}
    \definecolor{citecolor}{rgb}{.12,.54,.11}

    % ANSI colors
    \definecolor{ansi-black}{HTML}{3E424D}
    \definecolor{ansi-black-intense}{HTML}{282C36}
    \definecolor{ansi-red}{HTML}{E75C58}
    \definecolor{ansi-red-intense}{HTML}{B22B31}
    \definecolor{ansi-green}{HTML}{00A250}
    \definecolor{ansi-green-intense}{HTML}{007427}
    \definecolor{ansi-yellow}{HTML}{DDB62B}
    \definecolor{ansi-yellow-intense}{HTML}{B27D12}
    \definecolor{ansi-blue}{HTML}{208FFB}
    \definecolor{ansi-blue-intense}{HTML}{0065CA}
    \definecolor{ansi-magenta}{HTML}{D160C4}
    \definecolor{ansi-magenta-intense}{HTML}{A03196}
    \definecolor{ansi-cyan}{HTML}{60C6C8}
    \definecolor{ansi-cyan-intense}{HTML}{258F8F}
    \definecolor{ansi-white}{HTML}{C5C1B4}
    \definecolor{ansi-white-intense}{HTML}{A1A6B2}
    \definecolor{ansi-default-inverse-fg}{HTML}{FFFFFF}
    \definecolor{ansi-default-inverse-bg}{HTML}{000000}

    % common color for the border for error outputs.
    \definecolor{outerrorbackground}{HTML}{FFDFDF}

    % commands and environments needed by pandoc snippets
    % extracted from the output of `pandoc -s`
    \providecommand{\tightlist}{%
      \setlength{\itemsep}{0pt}\setlength{\parskip}{0pt}}
    \DefineVerbatimEnvironment{Highlighting}{Verbatim}{commandchars=\\\{\}}
    % Add ',fontsize=\small' for more characters per line
    \newenvironment{Shaded}{}{}
    \newcommand{\KeywordTok}[1]{\textcolor[rgb]{0.00,0.44,0.13}{\textbf{{#1}}}}
    \newcommand{\DataTypeTok}[1]{\textcolor[rgb]{0.56,0.13,0.00}{{#1}}}
    \newcommand{\DecValTok}[1]{\textcolor[rgb]{0.25,0.63,0.44}{{#1}}}
    \newcommand{\BaseNTok}[1]{\textcolor[rgb]{0.25,0.63,0.44}{{#1}}}
    \newcommand{\FloatTok}[1]{\textcolor[rgb]{0.25,0.63,0.44}{{#1}}}
    \newcommand{\CharTok}[1]{\textcolor[rgb]{0.25,0.44,0.63}{{#1}}}
    \newcommand{\StringTok}[1]{\textcolor[rgb]{0.25,0.44,0.63}{{#1}}}
    \newcommand{\CommentTok}[1]{\textcolor[rgb]{0.38,0.63,0.69}{\textit{{#1}}}}
    \newcommand{\OtherTok}[1]{\textcolor[rgb]{0.00,0.44,0.13}{{#1}}}
    \newcommand{\AlertTok}[1]{\textcolor[rgb]{1.00,0.00,0.00}{\textbf{{#1}}}}
    \newcommand{\FunctionTok}[1]{\textcolor[rgb]{0.02,0.16,0.49}{{#1}}}
    \newcommand{\RegionMarkerTok}[1]{{#1}}
    \newcommand{\ErrorTok}[1]{\textcolor[rgb]{1.00,0.00,0.00}{\textbf{{#1}}}}
    \newcommand{\NormalTok}[1]{{#1}}
    
    % Additional commands for more recent versions of Pandoc
    \newcommand{\ConstantTok}[1]{\textcolor[rgb]{0.53,0.00,0.00}{{#1}}}
    \newcommand{\SpecialCharTok}[1]{\textcolor[rgb]{0.25,0.44,0.63}{{#1}}}
    \newcommand{\VerbatimStringTok}[1]{\textcolor[rgb]{0.25,0.44,0.63}{{#1}}}
    \newcommand{\SpecialStringTok}[1]{\textcolor[rgb]{0.73,0.40,0.53}{{#1}}}
    \newcommand{\ImportTok}[1]{{#1}}
    \newcommand{\DocumentationTok}[1]{\textcolor[rgb]{0.73,0.13,0.13}{\textit{{#1}}}}
    \newcommand{\AnnotationTok}[1]{\textcolor[rgb]{0.38,0.63,0.69}{\textbf{\textit{{#1}}}}}
    \newcommand{\CommentVarTok}[1]{\textcolor[rgb]{0.38,0.63,0.69}{\textbf{\textit{{#1}}}}}
    \newcommand{\VariableTok}[1]{\textcolor[rgb]{0.10,0.09,0.49}{{#1}}}
    \newcommand{\ControlFlowTok}[1]{\textcolor[rgb]{0.00,0.44,0.13}{\textbf{{#1}}}}
    \newcommand{\OperatorTok}[1]{\textcolor[rgb]{0.40,0.40,0.40}{{#1}}}
    \newcommand{\BuiltInTok}[1]{{#1}}
    \newcommand{\ExtensionTok}[1]{{#1}}
    \newcommand{\PreprocessorTok}[1]{\textcolor[rgb]{0.74,0.48,0.00}{{#1}}}
    \newcommand{\AttributeTok}[1]{\textcolor[rgb]{0.49,0.56,0.16}{{#1}}}
    \newcommand{\InformationTok}[1]{\textcolor[rgb]{0.38,0.63,0.69}{\textbf{\textit{{#1}}}}}
    \newcommand{\WarningTok}[1]{\textcolor[rgb]{0.38,0.63,0.69}{\textbf{\textit{{#1}}}}}
    
    
    % Define a nice break command that doesn't care if a line doesn't already
    % exist.
    \def\br{\hspace*{\fill} \\* }
    % Math Jax compatibility definitions
    \def\gt{>}
    \def\lt{<}
    \let\Oldtex\TeX
    \let\Oldlatex\LaTeX
    \renewcommand{\TeX}{\textrm{\Oldtex}}
    \renewcommand{\LaTeX}{\textrm{\Oldlatex}}
    % Document parameters
    % Document title
    
    
    
    
    
% Pygments definitions
\makeatletter
\def\PY@reset{\let\PY@it=\relax \let\PY@bf=\relax%
    \let\PY@ul=\relax \let\PY@tc=\relax%
    \let\PY@bc=\relax \let\PY@ff=\relax}
\def\PY@tok#1{\csname PY@tok@#1\endcsname}
\def\PY@toks#1+{\ifx\relax#1\empty\else%
    \PY@tok{#1}\expandafter\PY@toks\fi}
\def\PY@do#1{\PY@bc{\PY@tc{\PY@ul{%
    \PY@it{\PY@bf{\PY@ff{#1}}}}}}}
\def\PY#1#2{\PY@reset\PY@toks#1+\relax+\PY@do{#2}}

\@namedef{PY@tok@w}{\def\PY@tc##1{\textcolor[rgb]{0.73,0.73,0.73}{##1}}}
\@namedef{PY@tok@c}{\let\PY@it=\textit\def\PY@tc##1{\textcolor[rgb]{0.25,0.50,0.50}{##1}}}
\@namedef{PY@tok@cp}{\def\PY@tc##1{\textcolor[rgb]{0.74,0.48,0.00}{##1}}}
\@namedef{PY@tok@k}{\let\PY@bf=\textbf\def\PY@tc##1{\textcolor[rgb]{0.00,0.50,0.00}{##1}}}
\@namedef{PY@tok@kp}{\def\PY@tc##1{\textcolor[rgb]{0.00,0.50,0.00}{##1}}}
\@namedef{PY@tok@kt}{\def\PY@tc##1{\textcolor[rgb]{0.69,0.00,0.25}{##1}}}
\@namedef{PY@tok@o}{\def\PY@tc##1{\textcolor[rgb]{0.40,0.40,0.40}{##1}}}
\@namedef{PY@tok@ow}{\let\PY@bf=\textbf\def\PY@tc##1{\textcolor[rgb]{0.67,0.13,1.00}{##1}}}
\@namedef{PY@tok@nb}{\def\PY@tc##1{\textcolor[rgb]{0.00,0.50,0.00}{##1}}}
\@namedef{PY@tok@nf}{\def\PY@tc##1{\textcolor[rgb]{0.00,0.00,1.00}{##1}}}
\@namedef{PY@tok@nc}{\let\PY@bf=\textbf\def\PY@tc##1{\textcolor[rgb]{0.00,0.00,1.00}{##1}}}
\@namedef{PY@tok@nn}{\let\PY@bf=\textbf\def\PY@tc##1{\textcolor[rgb]{0.00,0.00,1.00}{##1}}}
\@namedef{PY@tok@ne}{\let\PY@bf=\textbf\def\PY@tc##1{\textcolor[rgb]{0.82,0.25,0.23}{##1}}}
\@namedef{PY@tok@nv}{\def\PY@tc##1{\textcolor[rgb]{0.10,0.09,0.49}{##1}}}
\@namedef{PY@tok@no}{\def\PY@tc##1{\textcolor[rgb]{0.53,0.00,0.00}{##1}}}
\@namedef{PY@tok@nl}{\def\PY@tc##1{\textcolor[rgb]{0.63,0.63,0.00}{##1}}}
\@namedef{PY@tok@ni}{\let\PY@bf=\textbf\def\PY@tc##1{\textcolor[rgb]{0.60,0.60,0.60}{##1}}}
\@namedef{PY@tok@na}{\def\PY@tc##1{\textcolor[rgb]{0.49,0.56,0.16}{##1}}}
\@namedef{PY@tok@nt}{\let\PY@bf=\textbf\def\PY@tc##1{\textcolor[rgb]{0.00,0.50,0.00}{##1}}}
\@namedef{PY@tok@nd}{\def\PY@tc##1{\textcolor[rgb]{0.67,0.13,1.00}{##1}}}
\@namedef{PY@tok@s}{\def\PY@tc##1{\textcolor[rgb]{0.73,0.13,0.13}{##1}}}
\@namedef{PY@tok@sd}{\let\PY@it=\textit\def\PY@tc##1{\textcolor[rgb]{0.73,0.13,0.13}{##1}}}
\@namedef{PY@tok@si}{\let\PY@bf=\textbf\def\PY@tc##1{\textcolor[rgb]{0.73,0.40,0.53}{##1}}}
\@namedef{PY@tok@se}{\let\PY@bf=\textbf\def\PY@tc##1{\textcolor[rgb]{0.73,0.40,0.13}{##1}}}
\@namedef{PY@tok@sr}{\def\PY@tc##1{\textcolor[rgb]{0.73,0.40,0.53}{##1}}}
\@namedef{PY@tok@ss}{\def\PY@tc##1{\textcolor[rgb]{0.10,0.09,0.49}{##1}}}
\@namedef{PY@tok@sx}{\def\PY@tc##1{\textcolor[rgb]{0.00,0.50,0.00}{##1}}}
\@namedef{PY@tok@m}{\def\PY@tc##1{\textcolor[rgb]{0.40,0.40,0.40}{##1}}}
\@namedef{PY@tok@gh}{\let\PY@bf=\textbf\def\PY@tc##1{\textcolor[rgb]{0.00,0.00,0.50}{##1}}}
\@namedef{PY@tok@gu}{\let\PY@bf=\textbf\def\PY@tc##1{\textcolor[rgb]{0.50,0.00,0.50}{##1}}}
\@namedef{PY@tok@gd}{\def\PY@tc##1{\textcolor[rgb]{0.63,0.00,0.00}{##1}}}
\@namedef{PY@tok@gi}{\def\PY@tc##1{\textcolor[rgb]{0.00,0.63,0.00}{##1}}}
\@namedef{PY@tok@gr}{\def\PY@tc##1{\textcolor[rgb]{1.00,0.00,0.00}{##1}}}
\@namedef{PY@tok@ge}{\let\PY@it=\textit}
\@namedef{PY@tok@gs}{\let\PY@bf=\textbf}
\@namedef{PY@tok@gp}{\let\PY@bf=\textbf\def\PY@tc##1{\textcolor[rgb]{0.00,0.00,0.50}{##1}}}
\@namedef{PY@tok@go}{\def\PY@tc##1{\textcolor[rgb]{0.53,0.53,0.53}{##1}}}
\@namedef{PY@tok@gt}{\def\PY@tc##1{\textcolor[rgb]{0.00,0.27,0.87}{##1}}}
\@namedef{PY@tok@err}{\def\PY@bc##1{{\setlength{\fboxsep}{\string -\fboxrule}\fcolorbox[rgb]{1.00,0.00,0.00}{1,1,1}{\strut ##1}}}}
\@namedef{PY@tok@kc}{\let\PY@bf=\textbf\def\PY@tc##1{\textcolor[rgb]{0.00,0.50,0.00}{##1}}}
\@namedef{PY@tok@kd}{\let\PY@bf=\textbf\def\PY@tc##1{\textcolor[rgb]{0.00,0.50,0.00}{##1}}}
\@namedef{PY@tok@kn}{\let\PY@bf=\textbf\def\PY@tc##1{\textcolor[rgb]{0.00,0.50,0.00}{##1}}}
\@namedef{PY@tok@kr}{\let\PY@bf=\textbf\def\PY@tc##1{\textcolor[rgb]{0.00,0.50,0.00}{##1}}}
\@namedef{PY@tok@bp}{\def\PY@tc##1{\textcolor[rgb]{0.00,0.50,0.00}{##1}}}
\@namedef{PY@tok@fm}{\def\PY@tc##1{\textcolor[rgb]{0.00,0.00,1.00}{##1}}}
\@namedef{PY@tok@vc}{\def\PY@tc##1{\textcolor[rgb]{0.10,0.09,0.49}{##1}}}
\@namedef{PY@tok@vg}{\def\PY@tc##1{\textcolor[rgb]{0.10,0.09,0.49}{##1}}}
\@namedef{PY@tok@vi}{\def\PY@tc##1{\textcolor[rgb]{0.10,0.09,0.49}{##1}}}
\@namedef{PY@tok@vm}{\def\PY@tc##1{\textcolor[rgb]{0.10,0.09,0.49}{##1}}}
\@namedef{PY@tok@sa}{\def\PY@tc##1{\textcolor[rgb]{0.73,0.13,0.13}{##1}}}
\@namedef{PY@tok@sb}{\def\PY@tc##1{\textcolor[rgb]{0.73,0.13,0.13}{##1}}}
\@namedef{PY@tok@sc}{\def\PY@tc##1{\textcolor[rgb]{0.73,0.13,0.13}{##1}}}
\@namedef{PY@tok@dl}{\def\PY@tc##1{\textcolor[rgb]{0.73,0.13,0.13}{##1}}}
\@namedef{PY@tok@s2}{\def\PY@tc##1{\textcolor[rgb]{0.73,0.13,0.13}{##1}}}
\@namedef{PY@tok@sh}{\def\PY@tc##1{\textcolor[rgb]{0.73,0.13,0.13}{##1}}}
\@namedef{PY@tok@s1}{\def\PY@tc##1{\textcolor[rgb]{0.73,0.13,0.13}{##1}}}
\@namedef{PY@tok@mb}{\def\PY@tc##1{\textcolor[rgb]{0.40,0.40,0.40}{##1}}}
\@namedef{PY@tok@mf}{\def\PY@tc##1{\textcolor[rgb]{0.40,0.40,0.40}{##1}}}
\@namedef{PY@tok@mh}{\def\PY@tc##1{\textcolor[rgb]{0.40,0.40,0.40}{##1}}}
\@namedef{PY@tok@mi}{\def\PY@tc##1{\textcolor[rgb]{0.40,0.40,0.40}{##1}}}
\@namedef{PY@tok@il}{\def\PY@tc##1{\textcolor[rgb]{0.40,0.40,0.40}{##1}}}
\@namedef{PY@tok@mo}{\def\PY@tc##1{\textcolor[rgb]{0.40,0.40,0.40}{##1}}}
\@namedef{PY@tok@ch}{\let\PY@it=\textit\def\PY@tc##1{\textcolor[rgb]{0.25,0.50,0.50}{##1}}}
\@namedef{PY@tok@cm}{\let\PY@it=\textit\def\PY@tc##1{\textcolor[rgb]{0.25,0.50,0.50}{##1}}}
\@namedef{PY@tok@cpf}{\let\PY@it=\textit\def\PY@tc##1{\textcolor[rgb]{0.25,0.50,0.50}{##1}}}
\@namedef{PY@tok@c1}{\let\PY@it=\textit\def\PY@tc##1{\textcolor[rgb]{0.25,0.50,0.50}{##1}}}
\@namedef{PY@tok@cs}{\let\PY@it=\textit\def\PY@tc##1{\textcolor[rgb]{0.25,0.50,0.50}{##1}}}

\def\PYZbs{\char`\\}
\def\PYZus{\char`\_}
\def\PYZob{\char`\{}
\def\PYZcb{\char`\}}
\def\PYZca{\char`\^}
\def\PYZam{\char`\&}
\def\PYZlt{\char`\<}
\def\PYZgt{\char`\>}
\def\PYZsh{\char`\#}
\def\PYZpc{\char`\%}
\def\PYZdl{\char`\$}
\def\PYZhy{\char`\-}
\def\PYZsq{\char`\'}
\def\PYZdq{\char`\"}
\def\PYZti{\char`\~}
% for compatibility with earlier versions
\def\PYZat{@}
\def\PYZlb{[}
\def\PYZrb{]}
\makeatother


    % For linebreaks inside Verbatim environment from package fancyvrb. 
    \makeatletter
        \newbox\Wrappedcontinuationbox 
        \newbox\Wrappedvisiblespacebox 
        \newcommand*\Wrappedvisiblespace {\textcolor{red}{\textvisiblespace}} 
        \newcommand*\Wrappedcontinuationsymbol {\textcolor{red}{\llap{\tiny$\m@th\hookrightarrow$}}} 
        \newcommand*\Wrappedcontinuationindent {3ex } 
        \newcommand*\Wrappedafterbreak {\kern\Wrappedcontinuationindent\copy\Wrappedcontinuationbox} 
        % Take advantage of the already applied Pygments mark-up to insert 
        % potential linebreaks for TeX processing. 
        %        {, <, #, %, $, ' and ": go to next line. 
        %        _, }, ^, &, >, - and ~: stay at end of broken line. 
        % Use of \textquotesingle for straight quote. 
        \newcommand*\Wrappedbreaksatspecials {% 
            \def\PYGZus{\discretionary{\char`\_}{\Wrappedafterbreak}{\char`\_}}% 
            \def\PYGZob{\discretionary{}{\Wrappedafterbreak\char`\{}{\char`\{}}% 
            \def\PYGZcb{\discretionary{\char`\}}{\Wrappedafterbreak}{\char`\}}}% 
            \def\PYGZca{\discretionary{\char`\^}{\Wrappedafterbreak}{\char`\^}}% 
            \def\PYGZam{\discretionary{\char`\&}{\Wrappedafterbreak}{\char`\&}}% 
            \def\PYGZlt{\discretionary{}{\Wrappedafterbreak\char`\<}{\char`\<}}% 
            \def\PYGZgt{\discretionary{\char`\>}{\Wrappedafterbreak}{\char`\>}}% 
            \def\PYGZsh{\discretionary{}{\Wrappedafterbreak\char`\#}{\char`\#}}% 
            \def\PYGZpc{\discretionary{}{\Wrappedafterbreak\char`\%}{\char`\%}}% 
            \def\PYGZdl{\discretionary{}{\Wrappedafterbreak\char`\$}{\char`\$}}% 
            \def\PYGZhy{\discretionary{\char`\-}{\Wrappedafterbreak}{\char`\-}}% 
            \def\PYGZsq{\discretionary{}{\Wrappedafterbreak\textquotesingle}{\textquotesingle}}% 
            \def\PYGZdq{\discretionary{}{\Wrappedafterbreak\char`\"}{\char`\"}}% 
            \def\PYGZti{\discretionary{\char`\~}{\Wrappedafterbreak}{\char`\~}}% 
        } 
        % Some characters . , ; ? ! / are not pygmentized. 
        % This macro makes them "active" and they will insert potential linebreaks 
        \newcommand*\Wrappedbreaksatpunct {% 
            \lccode`\~`\.\lowercase{\def~}{\discretionary{\hbox{\char`\.}}{\Wrappedafterbreak}{\hbox{\char`\.}}}% 
            \lccode`\~`\,\lowercase{\def~}{\discretionary{\hbox{\char`\,}}{\Wrappedafterbreak}{\hbox{\char`\,}}}% 
            \lccode`\~`\;\lowercase{\def~}{\discretionary{\hbox{\char`\;}}{\Wrappedafterbreak}{\hbox{\char`\;}}}% 
            \lccode`\~`\:\lowercase{\def~}{\discretionary{\hbox{\char`\:}}{\Wrappedafterbreak}{\hbox{\char`\:}}}% 
            \lccode`\~`\?\lowercase{\def~}{\discretionary{\hbox{\char`\?}}{\Wrappedafterbreak}{\hbox{\char`\?}}}% 
            \lccode`\~`\!\lowercase{\def~}{\discretionary{\hbox{\char`\!}}{\Wrappedafterbreak}{\hbox{\char`\!}}}% 
            \lccode`\~`\/\lowercase{\def~}{\discretionary{\hbox{\char`\/}}{\Wrappedafterbreak}{\hbox{\char`\/}}}% 
            \catcode`\.\active
            \catcode`\,\active 
            \catcode`\;\active
            \catcode`\:\active
            \catcode`\?\active
            \catcode`\!\active
            \catcode`\/\active 
            \lccode`\~`\~ 	
        }
    \makeatother

    \let\OriginalVerbatim=\Verbatim
    \makeatletter
    \renewcommand{\Verbatim}[1][1]{%
        %\parskip\z@skip
        \sbox\Wrappedcontinuationbox {\Wrappedcontinuationsymbol}%
        \sbox\Wrappedvisiblespacebox {\FV@SetupFont\Wrappedvisiblespace}%
        \def\FancyVerbFormatLine ##1{\hsize\linewidth
            \vtop{\raggedright\hyphenpenalty\z@\exhyphenpenalty\z@
                \doublehyphendemerits\z@\finalhyphendemerits\z@
                \strut ##1\strut}%
        }%
        % If the linebreak is at a space, the latter will be displayed as visible
        % space at end of first line, and a continuation symbol starts next line.
        % Stretch/shrink are however usually zero for typewriter font.
        \def\FV@Space {%
            \nobreak\hskip\z@ plus\fontdimen3\font minus\fontdimen4\font
            \discretionary{\copy\Wrappedvisiblespacebox}{\Wrappedafterbreak}
            {\kern\fontdimen2\font}%
        }%
        
        % Allow breaks at special characters using \PYG... macros.
        \Wrappedbreaksatspecials
        % Breaks at punctuation characters . , ; ? ! and / need catcode=\active 	
        \OriginalVerbatim[#1,codes*=\Wrappedbreaksatpunct]%
    }
    \makeatother

    % Exact colors from NB
    \definecolor{incolor}{HTML}{303F9F}
    \definecolor{outcolor}{HTML}{D84315}
    \definecolor{cellborder}{HTML}{CFCFCF}
    \definecolor{cellbackground}{HTML}{F7F7F7}
    
    % prompt
    \makeatletter
    \newcommand{\boxspacing}{\kern\kvtcb@left@rule\kern\kvtcb@boxsep}
    \makeatother
    \newcommand{\prompt}[4]{
        {\ttfamily\llap{{\color{#2}[#3]:\hspace{3pt}#4}}\vspace{-\baselineskip}}
    }
    

    
    % Prevent overflowing lines due to hard-to-break entities
    \sloppy 
    % Setup hyperref package
    \hypersetup{
      breaklinks=true,  % so long urls are correctly broken across lines
      colorlinks=true,
      urlcolor=urlcolor,
      linkcolor=linkcolor,
      citecolor=citecolor,
      }
    % Slightly bigger margins than the latex defaults
    
    \geometry{verbose,tmargin=1in,bmargin=1in,lmargin=1in,rmargin=1in}

\title{Homotopy Continuation!}
\author{Miles Cochran-Branson}
\date{Friday, November 12}

\begin{document}
    
\maketitle
  
\tableofcontents
    
    \hypertarget{implementing-homotopy-continuation-in-julia}{%
\section{Implementing Homotopy Continuation in
Julia}\label{implementing-homotopy-continuation-in-julia}}

Say we're interested in solving the system of equations

\[
\begin{split}
f_1(x, y) & = (x^4 + y^4 − 1)(x^2 + y^2 − 2) + x^5y \\
f_2(x, y) & = x^2 + 2xy^2 - 2y^2 - \frac{1}{2}. 
\end{split}
\]

This can very simply be done using hometopy continuation and the package
\texttt{HometopyContinuation.jl} in \texttt{Julia} as follows

    \begin{tcolorbox}[breakable, size=fbox, boxrule=1pt, pad at break*=1mm,colback=cellbackground, colframe=cellborder]
\prompt{In}{incolor}{1}{\boxspacing}
\begin{Verbatim}[commandchars=\\\{\}]
\PY{c}{\PYZsh{} load packages}
\PY{k}{using} \PY{n}{HomotopyContinuation}
\end{Verbatim}
\end{tcolorbox}

    \begin{tcolorbox}[breakable, size=fbox, boxrule=1pt, pad at break*=1mm,colback=cellbackground, colframe=cellborder]
\prompt{In}{incolor}{2}{\boxspacing}
\begin{Verbatim}[commandchars=\\\{\}]
\PY{c}{\PYZsh{} declare variables x and y}
\PY{n+nd}{@var} \PY{n}{x} \PY{n}{y}
\PY{c}{\PYZsh{} define the polynomials}
\PY{n}{f₁} \PY{o}{=} \PY{p}{(}\PY{n}{x}\PY{o}{\PYZca{}}\PY{l+m+mi}{4} \PY{o}{+} \PY{n}{y}\PY{o}{\PYZca{}}\PY{l+m+mi}{4} \PY{o}{\PYZhy{}} \PY{l+m+mi}{1}\PY{p}{)} \PY{o}{*} \PY{p}{(}\PY{n}{x}\PY{o}{\PYZca{}}\PY{l+m+mi}{2} \PY{o}{+} \PY{n}{y}\PY{o}{\PYZca{}}\PY{l+m+mi}{2} \PY{o}{\PYZhy{}} \PY{l+m+mi}{2}\PY{p}{)} \PY{o}{+} \PY{n}{x}\PY{o}{\PYZca{}}\PY{l+m+mi}{5} \PY{o}{*} \PY{n}{y}
\PY{n}{f₂} \PY{o}{=} \PY{n}{x}\PY{o}{\PYZca{}}\PY{l+m+mi}{2} \PY{o}{+} \PY{l+m+mi}{2}\PY{o}{*}\PY{n}{x}\PY{o}{*}\PY{n}{y}\PY{o}{\PYZca{}}\PY{l+m+mi}{2} \PY{o}{\PYZhy{}} \PY{l+m+mi}{2}\PY{o}{*}\PY{n}{y}\PY{o}{\PYZca{}}\PY{l+m+mi}{2} \PY{o}{\PYZhy{}} \PY{l+m+mi}{1}\PY{o}{/}\PY{l+m+mi}{2}
\PY{n}{F} \PY{o}{=} \PY{n}{System}\PY{p}{(}\PY{p}{[}\PY{n}{f₁}\PY{p}{,} \PY{n}{f₂}\PY{p}{]}\PY{p}{)}
\PY{n}{result} \PY{o}{=} \PY{n}{solve}\PY{p}{(}\PY{n}{F}\PY{p}{)}
\end{Verbatim}
\end{tcolorbox}

    \begin{Verbatim}[commandchars=\\\{\}]
\textcolor{ansi-green}{Tracking 18 paths{\ldots} 100\%|██████████████████████████████| Time:
0:00:26}
\textcolor{ansi-blue}{  \# paths tracked:                  18}
\textcolor{ansi-blue}{  \# non-singular solutions (real):  18 (4)}
\textcolor{ansi-blue}{  \# singular endpoints (real):      0 (0)}
\textcolor{ansi-blue}{  \# total solutions (real):         18 (4)}
    \end{Verbatim}

            \begin{tcolorbox}[breakable, size=fbox, boxrule=.5pt, pad at break*=1mm, opacityfill=0]
\prompt{Out}{outcolor}{2}{\boxspacing}
\begin{Verbatim}[commandchars=\\\{\}]
Result with 18 solutions
========================
• 18 paths tracked
• 18 non-singular solutions (4 real)
• random\_seed: 0x6a9bef16
• start\_system: :polyhedral

\end{Verbatim}
\end{tcolorbox}
        
    and we can display the solutions

   \begin{tcolorbox}[breakable, size=fbox, boxrule=1pt, pad at break*=1mm,colback=cellbackground, colframe=cellborder]
\prompt{In}{incolor}{3}{\boxspacing}
\begin{Verbatim}[commandchars=\\\{\}]
\PY{n}{solutions}\PY{p}{(}\PY{n}{result}\PY{p}{)}
\end{Verbatim}
\end{tcolorbox}

    
    \begin{Verbatim}[commandchars=\\\{\}]
18-element Vector\{Vector\{ComplexF64\}\}:
 [0.9443571312488813 + 0.3118635017972661im, 0.32083818529039954 + 0.9677296009728283im]
 [0.8999179208471728 - 8.474091755303838e-33im, -1.2441827613422727 + 9.244463733058732e-33im]
 [1.0866676911062136 - 0.3290700669978769im, -0.24048106708661118 + 1.1350215823993672im]
 [0.06944255588971958 - 1.0734210145259255im, 0.2852544211796043 + 0.7076856100161802im]
 [0.9443571312488815 - 0.311863501797266im, 0.3208381852903996 - 0.9677296009728285im]
 [0.8708494909007279 + 0.03055030123533535im, 0.9716347867303131 + 0.21528324968394982im]
 [-1.0665536440076708 + 0.1424944286215267im, -0.3998429331325364 + 0.07871238147493467im]
 [-1.671421392838003 + 0.0im, 0.6552051858720408 + 7.52316384526264e-37im]
 [1.7132941375582666 - 0.5813863945698412im, 0.047514134780854124 + 1.252792951007688im]
 [1.0866676911062132 + 0.3290700669978767im, -0.24048106708661127 - 1.1350215823993668im]
 [0.8708494909007278 - 0.03055030123533537im, 0.9716347867303131 - 0.2152832496839498im]
 [0.8209788924342627 - 1.88079096131566e-36im, -0.6971326459489459 - 6.018531076210112e-36im]
 [0.06944255588971956 + 1.0734210145259255im, 0.28525442117960437 - 0.7076856100161802im]
 [0.07565391048031057 + 0.9487419814734106im, -0.24800445792173503 + 0.6838307098593375im]
 [-0.9368979667963299 + 3.5264830524668625e-38im, 0.312284081738601 + 2.82118644197349e-37im]
 [-1.0665536440076708 - 0.1424944286215266im, -0.3998429331325363 - 0.0787123814749346im]
 [0.07565391048031057 - 0.9487419814734106im, -0.24800445792173503 - 0.6838307098593376im]
 [1.7132941375582666 + 0.5813863945698413im, 0.04751413478085409 - 1.252792951007688im]
    \end{Verbatim}

    \hypertarget{implementing-our-own-method}{%
\subsection{Implementing our own
method}\label{implementing-our-own-method}}

In order to better understand this process we implement our own method
for solving. This involves first taking an Euler step, that is,
essentially solving a system of ODEs using the Euler method, and then
correcting our quasi-solution by using the Newton method.

The implementation starts by fixing a \emph{Homotopy}, \(H(x,t)\) which
can be expressed in the form

\[
    H(x,t) = t \cdot f(x) + (1 - t) \cdot g(x)
\]

where \(t \in [0,1] \subset \mathbb{R}\) , \(f(x)\) is our original
system, and \(g(x)\) is a simpler system of equations which we
\emph{know} the solution to. We then need to advance time by some
\(\Delta t\) and end up with the set of differential equations defined
by the Euler method as

\[
    H(x + \Delta x, t + \Delta t) \approx H(x,t) + \frac{\partial H}{\partial x} \Delta x
 + \frac{\partial H}{\partial t} \Delta t. 
 \]

Note that we know that we have \(H(x,t) = 0\) as we started with a
solution! Additionally, we require that
\(H(x + \Delta x, t + \Delta t) = 0\) and we know what our time step
\(\Delta t\) is. Thus, we need only solve the system

\[
    \frac{\partial H}{\partial x} \Delta x = -\frac{\partial H}{\partial t} \Delta t
\]

for \(\Delta x\). We then update our values of \(t\) and \(x\) by
incrementin by \(\Delta t\) and \(\Delta x\) and then implement Newton
to find values of \(t\) and \(x\) that lie on our curve. Finally, we
step over all \(\Delta t\) until we get to \(t = 1\) at which point
\(x\) will be a solution to our original system!

    \hypertarget{euler-step}{%
\subsubsection{Euler Step}\label{euler-step}}

We write a function to compute a Euler step.

    \begin{tcolorbox}[breakable, size=fbox, boxrule=1pt, pad at break*=1mm,colback=cellbackground, colframe=cellborder]
\prompt{In}{incolor}{4}{\boxspacing}
\begin{Verbatim}[commandchars=\\\{\}]
\PY{k}{function} \PY{n}{euler\PYZus{}step}\PY{p}{(}\PY{n}{H}\PY{o}{::}\PY{k+kt}{System}\PY{p}{,} \PY{n}{Δt}\PY{o}{::}\PY{k+kt}{Float64}\PY{p}{,} \PY{n}{x}\PY{o}{::}\PY{k+kt}{Vector}\PY{p}{,} \PY{n}{t}\PY{o}{::}\PY{k+kt}{Float64}\PY{p}{)}
    
    \PY{c}{\PYZsh{}make four element Vector}
    \PY{n}{xt} \PY{o}{=} \PY{n}{append!}\PY{p}{(}\PY{n}{x}\PY{p}{,} \PY{n}{t}\PY{p}{)}
    \PY{c}{\PYZsh{}get indices for vector subsetting}
    \PY{n}{index} \PY{o}{=} \PY{p}{[}\PY{p}{]}
    \PY{k}{for} \PY{n}{i} \PY{k}{in} \PY{l+m+mi}{2}\PY{o}{:}\PY{n}{length}\PY{p}{(}\PY{n}{x}\PY{p}{)}
        \PY{n}{append!}\PY{p}{(}\PY{n}{index}\PY{p}{,} \PY{n}{i}\PY{p}{)}
    \PY{k}{end}

    \PY{c}{\PYZsh{}make jacobian and separate out dH/dx and dH/dt}
    \PY{n}{J} \PY{o}{=} \PY{n}{jacobian}\PY{p}{(}\PY{n}{H}\PY{p}{,} \PY{n}{xt}\PY{p}{)}
    \PY{n}{∂H∂x} \PY{o}{=} \PY{n}{J}\PY{p}{[}\PY{o}{:}\PY{p}{,}\PY{n}{index}\PY{p}{]}
    \PY{n}{∂H∂t} \PY{o}{=} \PY{n}{J}\PY{p}{[}\PY{o}{:}\PY{p}{,}\PY{l+m+mi}{1}\PY{p}{]}

    \PY{c}{\PYZsh{}set up system and solve for Δx}
    \PY{n}{Δx} \PY{o}{=} \PY{n}{∂H∂x} \PY{o}{\PYZbs{}} \PY{o}{\PYZhy{}}\PY{p}{(}\PY{n}{∂H∂t} \PY{o}{*} \PY{n}{Δt}\PY{p}{)}

    \PY{c}{\PYZsh{}update values for x and t}
    \PY{n}{x} \PY{o}{=} \PY{n}{xt}\PY{p}{[}\PY{n}{index} \PY{o}{.\PYZhy{}} \PY{l+m+mi}{1}\PY{p}{]} \PY{o}{+} \PY{n}{Δx}
    \PY{n}{t} \PY{o}{=} \PY{n}{t} \PY{o}{+} \PY{n}{Δt}

    \PY{k}{return} \PY{n}{x}\PY{p}{,} \PY{n}{t}

\PY{k}{end}
\end{Verbatim}
\end{tcolorbox}

            \begin{tcolorbox}[breakable, size=fbox, boxrule=.5pt, pad at break*=1mm, opacityfill=0]
\prompt{Out}{outcolor}{4}{\boxspacing}
\begin{Verbatim}[commandchars=\\\{\}]
euler\_step (generic function with 1 method)
\end{Verbatim}
\end{tcolorbox}
        
    Notice that this function can take systems / vectors of any length and
is thus a general solution. Let's just make sure this works in practice
as follows

    \begin{tcolorbox}[breakable, size=fbox, boxrule=1pt, pad at break*=1mm,colback=cellbackground, colframe=cellborder]
\prompt{In}{incolor}{5}{\boxspacing}
\begin{Verbatim}[commandchars=\\\{\}]
\PY{n}{time} \PY{o}{=} \PY{l+m+mf}{0.0}\PY{p}{;}
\PY{n}{Δt} \PY{o}{=} \PY{l+m+mf}{0.1}\PY{p}{;} 

\PY{n}{x} \PY{o}{=} \PY{p}{[}\PY{l+m+mf}{1.0}\PY{p}{,} \PY{l+m+mf}{2.0}\PY{p}{]}

\PY{n+nd}{@var} \PY{n}{x1} \PY{n}{x2} \PY{n}{t}
\PY{n}{f1} \PY{o}{=} \PY{n}{t} \PY{o}{*} \PY{n}{x1}\PY{o}{\PYZca{}}\PY{l+m+mi}{2} \PY{o}{+} \PY{n}{x2}
\PY{n}{f2} \PY{o}{=} \PY{n}{x1} \PY{o}{+} \PY{n}{x2} \PY{o}{*} \PY{n}{t}

\PY{n}{F} \PY{o}{=} \PY{n}{System}\PY{p}{(}\PY{p}{[}\PY{n}{f1}\PY{p}{,} \PY{n}{f2}\PY{p}{]}\PY{p}{)}

\PY{n}{start}\PY{p}{,} \PY{n}{time} \PY{o}{=} \PY{n}{euler\PYZus{}step}\PY{p}{(}\PY{n}{F}\PY{p}{,} \PY{n}{Δt}\PY{p}{,} \PY{n}{x}\PY{p}{,} \PY{n}{time}\PY{p}{)}
\PY{n}{start}
\end{Verbatim}
\end{tcolorbox}

            \begin{tcolorbox}[breakable, size=fbox, boxrule=.5pt, pad at break*=1mm, opacityfill=0]
\prompt{Out}{outcolor}{5}{\boxspacing}
\begin{Verbatim}[commandchars=\\\{\}]
2-element Vector\{Float64\}:
 1.1333333333333333
 1.8666666666666667
\end{Verbatim}
\end{tcolorbox}
        
    and we see that this works to first order approximation.

    \hypertarget{newton-method}{%
\subsubsection{Newton Method}\label{newton-method}}

We write a function to implement the newton method. This implementation
requires \texttt{HomotopyContinuation.jl}.

    \begin{tcolorbox}[breakable, size=fbox, boxrule=1pt, pad at break*=1mm,colback=cellbackground, colframe=cellborder]
\prompt{In}{incolor}{6}{\boxspacing}
\begin{Verbatim}[commandchars=\\\{\}]
\PY{k}{function} \PY{n}{newton\PYZus{}solve}\PY{p}{(}\PY{n}{start}\PY{o}{::}\PY{k+kt}{Vector}\PY{p}{,} \PY{n}{F}\PY{o}{::}\PY{k+kt}{System}\PY{p}{,} \PY{n}{tolerance}\PY{o}{::}\PY{k+kt}{Float64} \PY{o}{=} \PY{l+m+mf}{1e\PYZhy{}15}\PY{p}{)}

    \PY{c}{\PYZsh{}evaluate initial point}
    \PY{n}{tol} \PY{o}{=} \PY{n}{evaluate}\PY{p}{(}\PY{n}{F}\PY{p}{,} \PY{n}{start}\PY{p}{)}

    \PY{c}{\PYZsh{}check if initial point is solution}
    \PY{n}{not\PYZus{}minimized} \PY{o}{=} \PY{n+nb}{true}
    \PY{k}{if} \PY{n}{tol} \PY{o}{==} \PY{n}{zeros}\PY{p}{(}\PY{n}{length}\PY{p}{(}\PY{n}{tol}\PY{p}{)}\PY{p}{)}
        \PY{n}{not\PYZus{}minimized} \PY{o}{=} \PY{n+nb}{false}
    \PY{k}{end}
    
    \PY{n}{i} \PY{o}{=} \PY{l+m+mi}{0}
    \PY{k}{while} \PY{p}{(}\PY{n}{not\PYZus{}minimized} \PY{o}{\PYZam{}\PYZam{}} \PY{n}{i} \PY{o}{\PYZlt{}} \PY{l+m+mi}{501}\PY{p}{)}
        \PY{c}{\PYZsh{}test for convergence }
        \PY{k}{if} \PY{n}{i} \PY{o}{==} \PY{l+m+mi}{500}
            \PY{n}{error}\PY{p}{(}\PY{l+s}{\PYZdq{}}\PY{l+s}{Algorithm did not converge!}\PY{l+s}{\PYZdq{}}\PY{p}{)}
        \PY{k}{end}

        \PY{c}{\PYZsh{}solve linear system}
        \PY{n}{J} \PY{o}{=} \PY{n}{jacobian}\PY{p}{(}\PY{n}{F}\PY{p}{,} \PY{n}{start}\PY{p}{)}
        \PY{n}{start\PYZus{}new} \PY{o}{=} \PY{n}{J} \PY{o}{\PYZbs{}} \PY{o}{\PYZhy{}}\PY{n}{evaluate}\PY{p}{(}\PY{n}{F}\PY{p}{,} \PY{n}{start}\PY{p}{)}

        \PY{c}{\PYZsh{}update values}
        \PY{k}{for} \PY{n}{i} \PY{k}{in} \PY{l+m+mi}{1}\PY{o}{:}\PY{n}{length}\PY{p}{(}\PY{n}{start}\PY{p}{)}
            \PY{n}{start}\PY{p}{[}\PY{n}{i}\PY{p}{]} \PY{o}{=} \PY{n}{start\PYZus{}new}\PY{p}{[}\PY{n}{i}\PY{p}{]} \PY{o}{+} \PY{n}{start}\PY{p}{[}\PY{n}{i}\PY{p}{]}
        \PY{k}{end}
        \PY{n}{tol} \PY{o}{=} \PY{n}{evaluate}\PY{p}{(}\PY{n}{F}\PY{p}{,} \PY{n}{start}\PY{p}{)}

        \PY{c}{\PYZsh{}test if tolerance is reached}
        \PY{n}{not\PYZus{}minimized} \PY{o}{=} \PY{n+nb}{false}
        \PY{k}{for} \PY{n}{i} \PY{k}{in} \PY{l+m+mi}{1}\PY{o}{:}\PY{n}{length}\PY{p}{(}\PY{n}{tol}\PY{p}{)}
            \PY{k}{if} \PY{n}{abs}\PY{p}{(}\PY{n}{tol}\PY{p}{[}\PY{n}{i}\PY{p}{]}\PY{p}{)} \PY{o}{\PYZgt{}} \PY{n}{tolerance}
                \PY{n}{not\PYZus{}minimized} \PY{o}{=} \PY{n+nb}{true}
            \PY{k}{end}
        \PY{k}{end}
        \PY{n}{i} \PY{o}{+=} \PY{l+m+mi}{1}
    \PY{k}{end}

    \PY{k}{return} \PY{n}{start}
\PY{k}{end}
\end{Verbatim}
\end{tcolorbox}

            \begin{tcolorbox}[breakable, size=fbox, boxrule=.5pt, pad at break*=1mm, opacityfill=0]
\prompt{Out}{outcolor}{6}{\boxspacing}
\begin{Verbatim}[commandchars=\\\{\}]
newton\_solve (generic function with 2 methods)
\end{Verbatim}
\end{tcolorbox}
        
    We note that this function also is general and can solve any problem. We
verify this works on the complex numbers by

    \begin{tcolorbox}[breakable, size=fbox, boxrule=1pt, pad at break*=1mm,colback=cellbackground, colframe=cellborder]
\prompt{In}{incolor}{7}{\boxspacing}
\begin{Verbatim}[commandchars=\\\{\}]
\PY{n+nd}{@var} \PY{n}{x} \PY{n}{y}
\PY{n}{f1} \PY{o}{=} \PY{n}{x}\PY{o}{\PYZca{}}\PY{l+m+mi}{2} \PY{o}{+} \PY{n}{y}\PY{o}{\PYZca{}}\PY{l+m+mi}{3} \PY{o}{\PYZhy{}} \PY{l+m+mi}{1}
\PY{n}{f2} \PY{o}{=} \PY{l+m+mi}{2}\PY{o}{*}\PY{n}{x} \PY{o}{\PYZhy{}} \PY{n}{y}
\PY{n}{F} \PY{o}{=} \PY{n}{System}\PY{p}{(}\PY{p}{[}\PY{n}{f1}\PY{p}{,} \PY{n}{f2}\PY{p}{]}\PY{p}{)}
\PY{n}{p} \PY{o}{=} \PY{n}{randn}\PY{p}{(}\PY{k+kt}{ComplexF64}\PY{p}{,} \PY{l+m+mi}{2}\PY{p}{)}

\PY{n}{newton\PYZus{}solve}\PY{p}{(}\PY{n}{p}\PY{p}{,} \PY{n}{F}\PY{p}{)}
\end{Verbatim}
\end{tcolorbox}

            \begin{tcolorbox}[breakable, size=fbox, boxrule=.5pt, pad at break*=1mm, opacityfill=0]
\prompt{Out}{outcolor}{7}{\boxspacing}
\begin{Verbatim}[commandchars=\\\{\}]
2-element Vector\{ComplexF64\}:
 -0.2933069324711349 - 0.4298373375930675im
 -0.5866138649422697 - 0.859674675186135im
\end{Verbatim}
\end{tcolorbox}
        
    and check our answer using \texttt{HomotopyContinuation.jl}

    \begin{tcolorbox}[breakable, size=fbox, boxrule=1pt, pad at break*=1mm,colback=cellbackground, colframe=cellborder]
\prompt{In}{incolor}{8}{\boxspacing}
\begin{Verbatim}[commandchars=\\\{\}]
\PY{n}{S} \PY{o}{=} \PY{n}{System}\PY{p}{(}\PY{p}{[}\PY{n}{f1}\PY{p}{,} \PY{n}{f2}\PY{p}{]}\PY{p}{)}
\PY{n}{result} \PY{o}{=} \PY{n}{solve}\PY{p}{(}\PY{n}{S}\PY{p}{)}
\end{Verbatim}
\end{tcolorbox}

    \begin{Verbatim}[commandchars=\\\{\}]
\textcolor{ansi-green}{Tracking 3 paths{\ldots} 100\%|███████████████████████████████| Time:
0:00:17}
\textcolor{ansi-blue}{  \# paths tracked:                  3}
\textcolor{ansi-blue}{  \# non-singular solutions (real):  3 (1)}
\textcolor{ansi-blue}{  \# singular endpoints (real):      0 (0)}
\textcolor{ansi-blue}{  \# total solutions (real):         3 (1)}
    \end{Verbatim}

            \begin{tcolorbox}[breakable, size=fbox, boxrule=.5pt, pad at break*=1mm, opacityfill=0]
\prompt{Out}{outcolor}{8}{\boxspacing}
\begin{Verbatim}[commandchars=\\\{\}]
Result with 3 solutions
=======================
• 3 paths tracked
• 3 non-singular solutions (1 real)
• random\_seed: 0x5093c109
• start\_system: :polyhedral

\end{Verbatim}
\end{tcolorbox}
        
    \begin{tcolorbox}[breakable, size=fbox, boxrule=1pt, pad at break*=1mm,colback=cellbackground, colframe=cellborder]
\prompt{In}{incolor}{9}{\boxspacing}
\begin{Verbatim}[commandchars=\\\{\}]
\PY{n}{solutions}\PY{p}{(}\PY{n}{result}\PY{p}{)}
\end{Verbatim}
\end{tcolorbox}

            \begin{tcolorbox}[breakable, size=fbox, boxrule=.5pt, pad at break*=1mm, opacityfill=0]
\prompt{Out}{outcolor}{9}{\boxspacing}
\begin{Verbatim}[commandchars=\\\{\}]
3-element Vector\{Vector\{ComplexF64\}\}:
 [-0.2933069324711348 - 0.42983733759306764im, -0.5866138649422696 -
0.8596746751861353im]
 [0.4616138649422696 - 2.407412430484045e-35im, 0.9232277298845392 -
4.81482486096809e-35im]
 [-0.2933069324711348 + 0.42983733759306764im, -0.5866138649422696 +
0.8596746751861353im]
\end{Verbatim}
\end{tcolorbox}
        
    \hypertarget{path-tracking-and-solving}{%
\subsubsection{Path Tracking and
Solving!}\label{path-tracking-and-solving}}

We finally combine the two functions and return to the original problem
to solve the system of two equations. We will use

\[
f(x, y) = \begin{pmatrix} f_1(x,y) \\ f_2(x, y) \end{pmatrix}
\]

Notice that we need some \(g(x)\). Let's use

\[
g(x, y) = \begin{pmatrix} x^6 - 1 \\ y^3 - 1 \end{pmatrix}
\]

which we \emph{know} has solutions

\[
    x = e^{ik\frac{2\pi}{6}}, k = 0,1,2,3,4,5
\]

and

\[
    y = e^{i\ell\frac{2\pi}{3}}, \ell = 0,1,2
\]

and we see that we will have 18 total solutions. Let's implement this
model and find the solution when \(k = 0, \ell = 0\) to see if we get a
solution that \texttt{HomotopyContinuation.jl} got.

    \begin{tcolorbox}[breakable, size=fbox, boxrule=1pt, pad at break*=1mm,colback=cellbackground, colframe=cellborder]
\prompt{In}{incolor}{10}{\boxspacing}
\begin{Verbatim}[commandchars=\\\{\}]
\PY{k}{function} \PY{n}{path\PYZus{}track}\PY{p}{(}\PY{n}{H}\PY{o}{::}\PY{k+kt}{System}\PY{p}{,} \PY{n}{start}\PY{o}{::}\PY{k+kt}{Vector}\PY{p}{,} \PY{n}{Δt}\PY{o}{::}\PY{k+kt}{Float64}\PY{p}{)}

    \PY{n}{time} \PY{o}{=} \PY{l+m+mf}{0.0}
    \PY{n}{tot\PYZus{}steps} \PY{o}{=} \PY{l+m+mi}{1} \PY{o}{/} \PY{n}{Δt}

    \PY{n}{i} \PY{o}{=} \PY{l+m+mi}{0}
    \PY{k}{while} \PY{n}{i} \PY{o}{\PYZlt{}} \PY{n}{tot\PYZus{}steps}
        \PY{n}{start}\PY{p}{,} \PY{n}{time} \PY{o}{=} \PY{n}{euler\PYZus{}step}\PY{p}{(}\PY{n}{H}\PY{p}{,} \PY{n}{Δt}\PY{p}{,} \PY{n}{start}\PY{p}{,} \PY{n}{time}\PY{p}{)}
        \PY{n}{start} \PY{o}{=} \PY{n}{newton\PYZus{}solve}\PY{p}{(}\PY{n}{start}\PY{p}{,} \PY{n}{System}\PY{p}{(}\PY{n}{evaluate}\PY{p}{(}\PY{n}{H}\PY{p}{,} \PY{p}{[}\PY{n}{time}\PY{p}{,} \PY{n}{x}\PY{p}{,} \PY{n}{y}\PY{p}{]}\PY{p}{)}\PY{p}{)}\PY{p}{,} \PY{l+m+mf}{1e\PYZhy{}12}\PY{p}{)}
        \PY{n}{i} \PY{o}{+=} \PY{l+m+mi}{1}
    \PY{k}{end}

    \PY{k}{return} \PY{n}{start}\PY{p}{;} 
\PY{k}{end}
\end{Verbatim}
\end{tcolorbox}

            \begin{tcolorbox}[breakable, size=fbox, boxrule=.5pt, pad at break*=1mm, opacityfill=0]
\prompt{Out}{outcolor}{10}{\boxspacing}
\begin{Verbatim}[commandchars=\\\{\}]
path\_track (generic function with 1 method)
\end{Verbatim}
\end{tcolorbox}
        
    Notice that this function is \emph{not} general and will only work for
two-variable systems in variables \(x\) and \(y\). This can easily be
modified by small changes to our code. We then evaluate our function

    \begin{tcolorbox}[breakable, size=fbox, boxrule=1pt, pad at break*=1mm,colback=cellbackground, colframe=cellborder]
\prompt{In}{incolor}{11}{\boxspacing}
\begin{Verbatim}[commandchars=\\\{\}]
\PY{c}{\PYZsh{}construct System}
\PY{n+nd}{@var} \PY{n}{x} \PY{n}{y} \PY{n}{t} 
\PY{n}{fg1} \PY{o}{=} \PY{p}{(}\PY{p}{(}\PY{n}{x}\PY{o}{\PYZca{}}\PY{l+m+mi}{4} \PY{o}{+} \PY{n}{y}\PY{o}{\PYZca{}}\PY{l+m+mi}{4} \PY{o}{\PYZhy{}} \PY{l+m+mi}{1}\PY{p}{)} \PY{o}{*} \PY{p}{(}\PY{n}{x}\PY{o}{\PYZca{}}\PY{l+m+mi}{2} \PY{o}{+} \PY{n}{y}\PY{o}{\PYZca{}}\PY{l+m+mi}{2} \PY{o}{\PYZhy{}} \PY{l+m+mi}{2}\PY{p}{)} \PY{o}{+} \PY{n}{x}\PY{o}{\PYZca{}}\PY{l+m+mi}{5} \PY{o}{*} \PY{n}{y}\PY{p}{)} \PY{o}{*} \PY{n}{t} \PY{o}{+} \PY{p}{(}\PY{l+m+mi}{1} \PY{o}{\PYZhy{}} \PY{n}{t}\PY{p}{)} \PY{o}{*} \PY{p}{(}\PY{n}{x}\PY{o}{\PYZca{}}\PY{l+m+mi}{6} \PY{o}{\PYZhy{}} \PY{l+m+mi}{1}\PY{p}{)}
\PY{n}{fg2} \PY{o}{=} \PY{p}{(}\PY{n}{x}\PY{o}{\PYZca{}}\PY{l+m+mi}{2} \PY{o}{+} \PY{l+m+mi}{2}\PY{o}{*}\PY{n}{x}\PY{o}{*}\PY{n}{y}\PY{o}{\PYZca{}}\PY{l+m+mi}{2} \PY{o}{\PYZhy{}} \PY{l+m+mi}{2}\PY{o}{*}\PY{n}{y}\PY{o}{\PYZca{}}\PY{l+m+mi}{2} \PY{o}{\PYZhy{}} \PY{l+m+mi}{1}\PY{o}{/}\PY{l+m+mi}{2}\PY{p}{)} \PY{o}{*} \PY{n}{t} \PY{o}{+} \PY{p}{(}\PY{l+m+mi}{1} \PY{o}{\PYZhy{}} \PY{n}{t}\PY{p}{)} \PY{o}{*} \PY{p}{(}\PY{n}{y}\PY{o}{\PYZca{}}\PY{l+m+mi}{3} \PY{o}{\PYZhy{}} \PY{l+m+mi}{1}\PY{p}{)}
\PY{n}{FS} \PY{o}{=} \PY{n}{System}\PY{p}{(}\PY{p}{[}\PY{n}{fg1}\PY{p}{,} \PY{n}{fg2}\PY{p}{]}\PY{p}{)}

\PY{n}{start\PYZus{}sol} \PY{o}{=} \PY{p}{[}\PY{n}{exp}\PY{p}{(}\PY{n+nb}{im}\PY{o}{*}\PY{l+m+mi}{0}\PY{o}{*}\PY{n+nb}{π}\PY{o}{/}\PY{l+m+mi}{6}\PY{p}{)}\PY{p}{,} \PY{n}{exp}\PY{p}{(}\PY{n+nb}{im}\PY{o}{*}\PY{l+m+mi}{0}\PY{o}{*}\PY{n+nb}{π}\PY{o}{/}\PY{l+m+mi}{3}\PY{p}{)}\PY{p}{]}

\PY{n}{Δt} \PY{o}{=} \PY{l+m+mf}{0.001}

\PY{n}{path\PYZus{}track}\PY{p}{(}\PY{n}{FS}\PY{p}{,} \PY{n}{start\PYZus{}sol}\PY{p}{,} \PY{n}{Δt}\PY{p}{)}
\end{Verbatim}
\end{tcolorbox}

            \begin{tcolorbox}[breakable, size=fbox, boxrule=.5pt, pad at break*=1mm, opacityfill=0]
\prompt{Out}{outcolor}{11}{\boxspacing}
\begin{Verbatim}[commandchars=\\\{\}]
2-element Vector\{ComplexF64\}:
  0.8209788924342625 - 1.2096476015712636e-18im
 -0.6971326459489465 - 3.947676315116146e-17im
\end{Verbatim}
\end{tcolorbox}
        
    Note that this is one of the points that \texttt{HomotopyContinuation}
got! Let's now find \emph{all} of the points by defining yet another
function that scans all the first points we give it and then solves.

    \begin{tcolorbox}[breakable, size=fbox, boxrule=1pt, pad at break*=1mm,colback=cellbackground, colframe=cellborder]
\prompt{In}{incolor}{12}{\boxspacing}
\begin{Verbatim}[commandchars=\\\{\}]
\PY{k}{function} \PY{n}{homotopy\PYZus{}solve}\PY{p}{(}\PY{n}{H}\PY{o}{::}\PY{k+kt}{System}\PY{p}{)}

    \PY{c}{\PYZsh{}set time step}
    \PY{n}{Δt} \PY{o}{=} \PY{l+m+mf}{0.001}

    \PY{c}{\PYZsh{}open array for appending solutions}
    \PY{n}{solutions} \PY{o}{=} \PY{p}{[}\PY{p}{]}

    \PY{c}{\PYZsh{}compute all solutions}
    \PY{c}{\PYZsh{}this only works for two equations which have degrees of 6 and 3 respectively }
    \PY{k}{for} \PY{n}{i} \PY{k}{in} \PY{l+m+mi}{0}\PY{o}{:}\PY{l+m+mi}{5}
        \PY{k}{for} \PY{n}{j} \PY{k}{in} \PY{l+m+mi}{0}\PY{o}{:}\PY{l+m+mi}{2}
            \PY{n}{start} \PY{o}{=} \PY{p}{[}\PY{n}{exp}\PY{p}{(}\PY{n}{i}\PY{o}{*}\PY{n+nb}{im}\PY{o}{*}\PY{l+m+mi}{2}\PY{o}{*}\PY{n+nb}{π}\PY{o}{/}\PY{l+m+mi}{6}\PY{p}{)}\PY{p}{,} \PY{n}{exp}\PY{p}{(}\PY{n}{j}\PY{o}{*}\PY{n+nb}{im}\PY{o}{*}\PY{l+m+mi}{2}\PY{o}{*}\PY{n+nb}{π}\PY{o}{/}\PY{l+m+mi}{3}\PY{p}{)}\PY{p}{]}
            \PY{n}{sol} \PY{o}{=} \PY{n}{path\PYZus{}track}\PY{p}{(}\PY{n}{H}\PY{p}{,} \PY{n}{start}\PY{p}{,} \PY{n}{Δt}\PY{p}{)}
            \PY{n}{push!}\PY{p}{(}\PY{n}{solutions}\PY{p}{,} \PY{n}{sol}\PY{p}{)}
        \PY{k}{end}
    \PY{k}{end}

    \PY{k}{return} \PY{n}{solutions}

\PY{k}{end}
\end{Verbatim}
\end{tcolorbox}

            \begin{tcolorbox}[breakable, size=fbox, boxrule=.5pt, pad at break*=1mm, opacityfill=0]
\prompt{Out}{outcolor}{12}{\boxspacing}
\begin{Verbatim}[commandchars=\\\{\}]
homotopy\_solve (generic function with 1 method)
\end{Verbatim}
\end{tcolorbox}
        
    Notice that, again, our function is \emph{not} general. Now it can only
solve the two variable case with polynomial degrees of six and three as
our problem is. This can be easily changed by small modifications of our
code. We then find the solutions by implementing our method.

    \begin{tcolorbox}[breakable, size=fbox, boxrule=1pt, pad at break*=1mm,colback=cellbackground, colframe=cellborder]
\prompt{In}{incolor}{13}{\boxspacing}
\begin{Verbatim}[commandchars=\\\{\}]
\PY{c}{\PYZsh{}solve them thangs}
\PY{n}{sol2} \PY{o}{=} \PY{n}{homotopy\PYZus{}solve}\PY{p}{(}\PY{n}{FS}\PY{p}{)}
\end{Verbatim}
\end{tcolorbox}

            \begin{tcolorbox}[breakable, size=fbox, boxrule=.5pt, pad at break*=1mm, opacityfill=0]
\prompt{Out}{outcolor}{13}{\boxspacing}
\begin{Verbatim}[commandchars=\\\{\}]
18-element Vector\{Any\}:
 ComplexF64[0.8209788924342625 - 1.2096476015712636e-18im, -0.6971326459489465 -
3.947676315116146e-17im]
 ComplexF64[0.9443571312488817 + 0.3118635017972665im, 0.3208381852903997 +
0.9677296009728291im]
 ComplexF64[0.9443571312488817 - 0.3118635017972665im, 0.3208381852903997 -
0.9677296009728291im]
 ComplexF64[1.713294137558267 + 0.5813863945698415im, 0.04751413478085409 -
1.2527929510076878im]
 ComplexF64[0.07565391048031049 + 0.9487419814734106im, -0.24800445792173492 +
0.6838307098593375im]
 ComplexF64[0.8999179208471724 + 1.577903428960801e-20im, -1.244182761342273 -
2.0837887478012815e-18im]
 ComplexF64[0.06944255588971956 + 1.0734210145259258im, 0.2852544211796043 -
0.7076856100161801im]
 ComplexF64[-1.0665536440076715 + 0.1424944286215246im, -0.3998429331325369 +
0.07871238147493335im]
 ComplexF64[1.0866676911062139 + 0.32907006699787644im, -0.2404810670866118 -
1.1350215823993675im]
 ComplexF64[0.8209788924342625 - 1.2096476015712636e-18im, -0.6971326459489465 -
3.947676315116146e-17im]
 ComplexF64[-1.6714213928380037 - 1.0186441737974984e-16im, 0.655205185872041 -
1.3262203461352321e-17im]
 ComplexF64[-1.6714213928380037 + 1.0186441737975248e-16im, 0.655205185872041 +
1.3262203461352648e-17im]
 ComplexF64[0.06944255588971956 - 1.0734210145259258im, 0.2852544211796043 +
0.7076856100161801im]
 ComplexF64[1.0866676911062139 - 0.32907006699787644im, -0.2404810670866118 +
1.1350215823993675im]
 ComplexF64[-1.0665536440076715 - 0.1424944286215246im, -0.3998429331325369 -
0.07871238147493335im]
 ComplexF64[1.713294137558267 - 0.5813863945698415im, 0.04751413478085409 +
1.2527929510076878im]
 ComplexF64[0.8999179208471724 - 1.577903428960801e-20im, -1.244182761342273 +
2.0837887478012815e-18im]
 ComplexF64[0.07565391048031049 - 0.9487419814734106im, -0.24800445792173492 -
0.6838307098593375im]
\end{Verbatim}
\end{tcolorbox}
        
    Notice that these are the same solutions that
\texttt{HomotopyContinuation} got! Discrepancies arrise for values that
are effectively zero, otherwise our answer is exactly the same as the
above solution.

\hypertarget{conclusions}{%
\section{Conclusions}\label{conclusions}}

We were able to implement concepts of Homotopy Continuation to find all
of the solutions to our system of equations. Of course, making our
functions general, i.e.~able to solve \emph{any} system of equations is
possible, but putting the work in to implement this is rather pointless
as \texttt{HomotopyContinuation.jl} does a fantastic job of solving any
system of equations in just a few lines of code (and we're already using
functions from \texttt{HomotopyContinuation.jl} for our implementation).
The above implementation shows, however, how powerful techniques in
Homotopy Continuation can be in solving polynomial equations and how
easy these methods are to implement.


    % Add a bibliography block to the postdoc
    
    
    
\end{document}
