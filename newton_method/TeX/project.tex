\documentclass[11pt]{article}

\title{Exploring the Newton-Ralphson Method}
\author{Miles Cochran-Branson}
\date{Friday, November 5}
    
\usepackage{newunicodechar}
    \usepackage{tikzsymbols}  % provides \Snowman
    \newunicodechar{λ}{$\lambda$}

    \usepackage[breakable]{tcolorbox}
    \usepackage{parskip} % Stop auto-indenting (to mimic markdown behaviour)
    
    \usepackage{iftex}
    \ifPDFTeX
    	\usepackage[T1]{fontenc}
    	\usepackage{mathpazo}
    \else
    	\usepackage{fontspec}
    \fi
    

    % Basic figure setup, for now with no caption control since it's done
    % automatically by Pandoc (which extracts ![](path) syntax from Markdown).
    \usepackage{graphicx}
    % Maintain compatibility with old templates. Remove in nbconvert 6.0
    \let\Oldincludegraphics\includegraphics
    % Ensure that by default, figures have no caption (until we provide a
    % proper Figure object with a Caption API and a way to capture that
    % in the conversion process - todo).
    \usepackage{caption}
    \DeclareCaptionFormat{nocaption}{}
    \captionsetup{format=nocaption,aboveskip=0pt,belowskip=0pt}

    \usepackage{float}
    \floatplacement{figure}{H} % forces figures to be placed at the correct location
    \usepackage{xcolor} % Allow colors to be defined
    \usepackage{enumerate} % Needed for markdown enumerations to work
    \usepackage{geometry} % Used to adjust the document margins
    \usepackage{amsmath} % Equations
    \usepackage{amssymb} % Equations
    \usepackage{textcomp} % defines textquotesingle
    % Hack from http://tex.stackexchange.com/a/47451/13684:
    \AtBeginDocument{%
        \def\PYZsq{\textquotesingle}% Upright quotes in Pygmentized code
    }
    \usepackage{upquote} % Upright quotes for verbatim code
    \usepackage{eurosym} % defines \euro
    \usepackage[mathletters]{ucs} % Extended unicode (utf-8) support
    \usepackage{fancyvrb} % verbatim replacement that allows latex
    \usepackage{grffile} % extends the file name processing of package graphics 
                         % to support a larger range
    \makeatletter % fix for old versions of grffile with XeLaTeX
    \@ifpackagelater{grffile}{2019/11/01}
    {
      % Do nothing on new versions
    }
    {
      \def\Gread@@xetex#1{%
        \IfFileExists{"\Gin@base".bb}%
        {\Gread@eps{\Gin@base.bb}}%
        {\Gread@@xetex@aux#1}%
      }
    }
    \makeatother
    \usepackage[Export]{adjustbox} % Used to constrain images to a maximum size
    \adjustboxset{max size={0.9\linewidth}{0.9\paperheight}}

    % The hyperref package gives us a pdf with properly built
    % internal navigation ('pdf bookmarks' for the table of contents,
    % internal cross-reference links, web links for URLs, etc.)
    \usepackage{hyperref}
    % The default LaTeX title has an obnoxious amount of whitespace. By default,
    % titling removes some of it. It also provides customization options.
    \usepackage{titling}
    \usepackage{longtable} % longtable support required by pandoc >1.10
    \usepackage{booktabs}  % table support for pandoc > 1.12.2
    \usepackage[inline]{enumitem} % IRkernel/repr support (it uses the enumerate* environment)
    \usepackage[normalem]{ulem} % ulem is needed to support strikethroughs (\sout)
                                % normalem makes italics be italics, not underlines
    \usepackage{mathrsfs}
    

    
    % Colors for the hyperref package
    \definecolor{urlcolor}{rgb}{0,.145,.698}
    \definecolor{linkcolor}{rgb}{.71,0.21,0.01}
    \definecolor{citecolor}{rgb}{.12,.54,.11}

    % ANSI colors
    \definecolor{ansi-black}{HTML}{3E424D}
    \definecolor{ansi-black-intense}{HTML}{282C36}
    \definecolor{ansi-red}{HTML}{E75C58}
    \definecolor{ansi-red-intense}{HTML}{B22B31}
    \definecolor{ansi-green}{HTML}{00A250}
    \definecolor{ansi-green-intense}{HTML}{007427}
    \definecolor{ansi-yellow}{HTML}{DDB62B}
    \definecolor{ansi-yellow-intense}{HTML}{B27D12}
    \definecolor{ansi-blue}{HTML}{208FFB}
    \definecolor{ansi-blue-intense}{HTML}{0065CA}
    \definecolor{ansi-magenta}{HTML}{D160C4}
    \definecolor{ansi-magenta-intense}{HTML}{A03196}
    \definecolor{ansi-cyan}{HTML}{60C6C8}
    \definecolor{ansi-cyan-intense}{HTML}{258F8F}
    \definecolor{ansi-white}{HTML}{C5C1B4}
    \definecolor{ansi-white-intense}{HTML}{A1A6B2}
    \definecolor{ansi-default-inverse-fg}{HTML}{FFFFFF}
    \definecolor{ansi-default-inverse-bg}{HTML}{000000}

    % common color for the border for error outputs.
    \definecolor{outerrorbackground}{HTML}{FFDFDF}

    % commands and environments needed by pandoc snippets
    % extracted from the output of `pandoc -s`
    \providecommand{\tightlist}{%
      \setlength{\itemsep}{0pt}\setlength{\parskip}{0pt}}
    \DefineVerbatimEnvironment{Highlighting}{Verbatim}{commandchars=\\\{\}}
    % Add ',fontsize=\small' for more characters per line
    \newenvironment{Shaded}{}{}
    \newcommand{\KeywordTok}[1]{\textcolor[rgb]{0.00,0.44,0.13}{\textbf{{#1}}}}
    \newcommand{\DataTypeTok}[1]{\textcolor[rgb]{0.56,0.13,0.00}{{#1}}}
    \newcommand{\DecValTok}[1]{\textcolor[rgb]{0.25,0.63,0.44}{{#1}}}
    \newcommand{\BaseNTok}[1]{\textcolor[rgb]{0.25,0.63,0.44}{{#1}}}
    \newcommand{\FloatTok}[1]{\textcolor[rgb]{0.25,0.63,0.44}{{#1}}}
    \newcommand{\CharTok}[1]{\textcolor[rgb]{0.25,0.44,0.63}{{#1}}}
    \newcommand{\StringTok}[1]{\textcolor[rgb]{0.25,0.44,0.63}{{#1}}}
    \newcommand{\CommentTok}[1]{\textcolor[rgb]{0.38,0.63,0.69}{\textit{{#1}}}}
    \newcommand{\OtherTok}[1]{\textcolor[rgb]{0.00,0.44,0.13}{{#1}}}
    \newcommand{\AlertTok}[1]{\textcolor[rgb]{1.00,0.00,0.00}{\textbf{{#1}}}}
    \newcommand{\FunctionTok}[1]{\textcolor[rgb]{0.02,0.16,0.49}{{#1}}}
    \newcommand{\RegionMarkerTok}[1]{{#1}}
    \newcommand{\ErrorTok}[1]{\textcolor[rgb]{1.00,0.00,0.00}{\textbf{{#1}}}}
    \newcommand{\NormalTok}[1]{{#1}}
    
    % Additional commands for more recent versions of Pandoc
    \newcommand{\ConstantTok}[1]{\textcolor[rgb]{0.53,0.00,0.00}{{#1}}}
    \newcommand{\SpecialCharTok}[1]{\textcolor[rgb]{0.25,0.44,0.63}{{#1}}}
    \newcommand{\VerbatimStringTok}[1]{\textcolor[rgb]{0.25,0.44,0.63}{{#1}}}
    \newcommand{\SpecialStringTok}[1]{\textcolor[rgb]{0.73,0.40,0.53}{{#1}}}
    \newcommand{\ImportTok}[1]{{#1}}
    \newcommand{\DocumentationTok}[1]{\textcolor[rgb]{0.73,0.13,0.13}{\textit{{#1}}}}
    \newcommand{\AnnotationTok}[1]{\textcolor[rgb]{0.38,0.63,0.69}{\textbf{\textit{{#1}}}}}
    \newcommand{\CommentVarTok}[1]{\textcolor[rgb]{0.38,0.63,0.69}{\textbf{\textit{{#1}}}}}
    \newcommand{\VariableTok}[1]{\textcolor[rgb]{0.10,0.09,0.49}{{#1}}}
    \newcommand{\ControlFlowTok}[1]{\textcolor[rgb]{0.00,0.44,0.13}{\textbf{{#1}}}}
    \newcommand{\OperatorTok}[1]{\textcolor[rgb]{0.40,0.40,0.40}{{#1}}}
    \newcommand{\BuiltInTok}[1]{{#1}}
    \newcommand{\ExtensionTok}[1]{{#1}}
    \newcommand{\PreprocessorTok}[1]{\textcolor[rgb]{0.74,0.48,0.00}{{#1}}}
    \newcommand{\AttributeTok}[1]{\textcolor[rgb]{0.49,0.56,0.16}{{#1}}}
    \newcommand{\InformationTok}[1]{\textcolor[rgb]{0.38,0.63,0.69}{\textbf{\textit{{#1}}}}}
    \newcommand{\WarningTok}[1]{\textcolor[rgb]{0.38,0.63,0.69}{\textbf{\textit{{#1}}}}}
    
    
    % Define a nice break command that doesn't care if a line doesn't already
    % exist.
    \def\br{\hspace*{\fill} \\* }
    % Math Jax compatibility definitions
    \def\gt{>}
    \def\lt{<}
    \let\Oldtex\TeX
    \let\Oldlatex\LaTeX
    \renewcommand{\TeX}{\textrm{\Oldtex}}
    \renewcommand{\LaTeX}{\textrm{\Oldlatex}}
    % Document parameters
    % Document title
    
    
    
    
    
% Pygments definitions
\makeatletter
\def\PY@reset{\let\PY@it=\relax \let\PY@bf=\relax%
    \let\PY@ul=\relax \let\PY@tc=\relax%
    \let\PY@bc=\relax \let\PY@ff=\relax}
\def\PY@tok#1{\csname PY@tok@#1\endcsname}
\def\PY@toks#1+{\ifx\relax#1\empty\else%
    \PY@tok{#1}\expandafter\PY@toks\fi}
\def\PY@do#1{\PY@bc{\PY@tc{\PY@ul{%
    \PY@it{\PY@bf{\PY@ff{#1}}}}}}}
\def\PY#1#2{\PY@reset\PY@toks#1+\relax+\PY@do{#2}}

\@namedef{PY@tok@w}{\def\PY@tc##1{\textcolor[rgb]{0.73,0.73,0.73}{##1}}}
\@namedef{PY@tok@c}{\let\PY@it=\textit\def\PY@tc##1{\textcolor[rgb]{0.25,0.50,0.50}{##1}}}
\@namedef{PY@tok@cp}{\def\PY@tc##1{\textcolor[rgb]{0.74,0.48,0.00}{##1}}}
\@namedef{PY@tok@k}{\let\PY@bf=\textbf\def\PY@tc##1{\textcolor[rgb]{0.00,0.50,0.00}{##1}}}
\@namedef{PY@tok@kp}{\def\PY@tc##1{\textcolor[rgb]{0.00,0.50,0.00}{##1}}}
\@namedef{PY@tok@kt}{\def\PY@tc##1{\textcolor[rgb]{0.69,0.00,0.25}{##1}}}
\@namedef{PY@tok@o}{\def\PY@tc##1{\textcolor[rgb]{0.40,0.40,0.40}{##1}}}
\@namedef{PY@tok@ow}{\let\PY@bf=\textbf\def\PY@tc##1{\textcolor[rgb]{0.67,0.13,1.00}{##1}}}
\@namedef{PY@tok@nb}{\def\PY@tc##1{\textcolor[rgb]{0.00,0.50,0.00}{##1}}}
\@namedef{PY@tok@nf}{\def\PY@tc##1{\textcolor[rgb]{0.00,0.00,1.00}{##1}}}
\@namedef{PY@tok@nc}{\let\PY@bf=\textbf\def\PY@tc##1{\textcolor[rgb]{0.00,0.00,1.00}{##1}}}
\@namedef{PY@tok@nn}{\let\PY@bf=\textbf\def\PY@tc##1{\textcolor[rgb]{0.00,0.00,1.00}{##1}}}
\@namedef{PY@tok@ne}{\let\PY@bf=\textbf\def\PY@tc##1{\textcolor[rgb]{0.82,0.25,0.23}{##1}}}
\@namedef{PY@tok@nv}{\def\PY@tc##1{\textcolor[rgb]{0.10,0.09,0.49}{##1}}}
\@namedef{PY@tok@no}{\def\PY@tc##1{\textcolor[rgb]{0.53,0.00,0.00}{##1}}}
\@namedef{PY@tok@nl}{\def\PY@tc##1{\textcolor[rgb]{0.63,0.63,0.00}{##1}}}
\@namedef{PY@tok@ni}{\let\PY@bf=\textbf\def\PY@tc##1{\textcolor[rgb]{0.60,0.60,0.60}{##1}}}
\@namedef{PY@tok@na}{\def\PY@tc##1{\textcolor[rgb]{0.49,0.56,0.16}{##1}}}
\@namedef{PY@tok@nt}{\let\PY@bf=\textbf\def\PY@tc##1{\textcolor[rgb]{0.00,0.50,0.00}{##1}}}
\@namedef{PY@tok@nd}{\def\PY@tc##1{\textcolor[rgb]{0.67,0.13,1.00}{##1}}}
\@namedef{PY@tok@s}{\def\PY@tc##1{\textcolor[rgb]{0.73,0.13,0.13}{##1}}}
\@namedef{PY@tok@sd}{\let\PY@it=\textit\def\PY@tc##1{\textcolor[rgb]{0.73,0.13,0.13}{##1}}}
\@namedef{PY@tok@si}{\let\PY@bf=\textbf\def\PY@tc##1{\textcolor[rgb]{0.73,0.40,0.53}{##1}}}
\@namedef{PY@tok@se}{\let\PY@bf=\textbf\def\PY@tc##1{\textcolor[rgb]{0.73,0.40,0.13}{##1}}}
\@namedef{PY@tok@sr}{\def\PY@tc##1{\textcolor[rgb]{0.73,0.40,0.53}{##1}}}
\@namedef{PY@tok@ss}{\def\PY@tc##1{\textcolor[rgb]{0.10,0.09,0.49}{##1}}}
\@namedef{PY@tok@sx}{\def\PY@tc##1{\textcolor[rgb]{0.00,0.50,0.00}{##1}}}
\@namedef{PY@tok@m}{\def\PY@tc##1{\textcolor[rgb]{0.40,0.40,0.40}{##1}}}
\@namedef{PY@tok@gh}{\let\PY@bf=\textbf\def\PY@tc##1{\textcolor[rgb]{0.00,0.00,0.50}{##1}}}
\@namedef{PY@tok@gu}{\let\PY@bf=\textbf\def\PY@tc##1{\textcolor[rgb]{0.50,0.00,0.50}{##1}}}
\@namedef{PY@tok@gd}{\def\PY@tc##1{\textcolor[rgb]{0.63,0.00,0.00}{##1}}}
\@namedef{PY@tok@gi}{\def\PY@tc##1{\textcolor[rgb]{0.00,0.63,0.00}{##1}}}
\@namedef{PY@tok@gr}{\def\PY@tc##1{\textcolor[rgb]{1.00,0.00,0.00}{##1}}}
\@namedef{PY@tok@ge}{\let\PY@it=\textit}
\@namedef{PY@tok@gs}{\let\PY@bf=\textbf}
\@namedef{PY@tok@gp}{\let\PY@bf=\textbf\def\PY@tc##1{\textcolor[rgb]{0.00,0.00,0.50}{##1}}}
\@namedef{PY@tok@go}{\def\PY@tc##1{\textcolor[rgb]{0.53,0.53,0.53}{##1}}}
\@namedef{PY@tok@gt}{\def\PY@tc##1{\textcolor[rgb]{0.00,0.27,0.87}{##1}}}
\@namedef{PY@tok@err}{\def\PY@bc##1{{\setlength{\fboxsep}{\string -\fboxrule}\fcolorbox[rgb]{1.00,0.00,0.00}{1,1,1}{\strut ##1}}}}
\@namedef{PY@tok@kc}{\let\PY@bf=\textbf\def\PY@tc##1{\textcolor[rgb]{0.00,0.50,0.00}{##1}}}
\@namedef{PY@tok@kd}{\let\PY@bf=\textbf\def\PY@tc##1{\textcolor[rgb]{0.00,0.50,0.00}{##1}}}
\@namedef{PY@tok@kn}{\let\PY@bf=\textbf\def\PY@tc##1{\textcolor[rgb]{0.00,0.50,0.00}{##1}}}
\@namedef{PY@tok@kr}{\let\PY@bf=\textbf\def\PY@tc##1{\textcolor[rgb]{0.00,0.50,0.00}{##1}}}
\@namedef{PY@tok@bp}{\def\PY@tc##1{\textcolor[rgb]{0.00,0.50,0.00}{##1}}}
\@namedef{PY@tok@fm}{\def\PY@tc##1{\textcolor[rgb]{0.00,0.00,1.00}{##1}}}
\@namedef{PY@tok@vc}{\def\PY@tc##1{\textcolor[rgb]{0.10,0.09,0.49}{##1}}}
\@namedef{PY@tok@vg}{\def\PY@tc##1{\textcolor[rgb]{0.10,0.09,0.49}{##1}}}
\@namedef{PY@tok@vi}{\def\PY@tc##1{\textcolor[rgb]{0.10,0.09,0.49}{##1}}}
\@namedef{PY@tok@vm}{\def\PY@tc##1{\textcolor[rgb]{0.10,0.09,0.49}{##1}}}
\@namedef{PY@tok@sa}{\def\PY@tc##1{\textcolor[rgb]{0.73,0.13,0.13}{##1}}}
\@namedef{PY@tok@sb}{\def\PY@tc##1{\textcolor[rgb]{0.73,0.13,0.13}{##1}}}
\@namedef{PY@tok@sc}{\def\PY@tc##1{\textcolor[rgb]{0.73,0.13,0.13}{##1}}}
\@namedef{PY@tok@dl}{\def\PY@tc##1{\textcolor[rgb]{0.73,0.13,0.13}{##1}}}
\@namedef{PY@tok@s2}{\def\PY@tc##1{\textcolor[rgb]{0.73,0.13,0.13}{##1}}}
\@namedef{PY@tok@sh}{\def\PY@tc##1{\textcolor[rgb]{0.73,0.13,0.13}{##1}}}
\@namedef{PY@tok@s1}{\def\PY@tc##1{\textcolor[rgb]{0.73,0.13,0.13}{##1}}}
\@namedef{PY@tok@mb}{\def\PY@tc##1{\textcolor[rgb]{0.40,0.40,0.40}{##1}}}
\@namedef{PY@tok@mf}{\def\PY@tc##1{\textcolor[rgb]{0.40,0.40,0.40}{##1}}}
\@namedef{PY@tok@mh}{\def\PY@tc##1{\textcolor[rgb]{0.40,0.40,0.40}{##1}}}
\@namedef{PY@tok@mi}{\def\PY@tc##1{\textcolor[rgb]{0.40,0.40,0.40}{##1}}}
\@namedef{PY@tok@il}{\def\PY@tc##1{\textcolor[rgb]{0.40,0.40,0.40}{##1}}}
\@namedef{PY@tok@mo}{\def\PY@tc##1{\textcolor[rgb]{0.40,0.40,0.40}{##1}}}
\@namedef{PY@tok@ch}{\let\PY@it=\textit\def\PY@tc##1{\textcolor[rgb]{0.25,0.50,0.50}{##1}}}
\@namedef{PY@tok@cm}{\let\PY@it=\textit\def\PY@tc##1{\textcolor[rgb]{0.25,0.50,0.50}{##1}}}
\@namedef{PY@tok@cpf}{\let\PY@it=\textit\def\PY@tc##1{\textcolor[rgb]{0.25,0.50,0.50}{##1}}}
\@namedef{PY@tok@c1}{\let\PY@it=\textit\def\PY@tc##1{\textcolor[rgb]{0.25,0.50,0.50}{##1}}}
\@namedef{PY@tok@cs}{\let\PY@it=\textit\def\PY@tc##1{\textcolor[rgb]{0.25,0.50,0.50}{##1}}}

\def\PYZbs{\char`\\}
\def\PYZus{\char`\_}
\def\PYZob{\char`\{}
\def\PYZcb{\char`\}}
\def\PYZca{\char`\^}
\def\PYZam{\char`\&}
\def\PYZlt{\char`\<}
\def\PYZgt{\char`\>}
\def\PYZsh{\char`\#}
\def\PYZpc{\char`\%}
\def\PYZdl{\char`\$}
\def\PYZhy{\char`\-}
\def\PYZsq{\char`\'}
\def\PYZdq{\char`\"}
\def\PYZti{\char`\~}
% for compatibility with earlier versions
\def\PYZat{@}
\def\PYZlb{[}
\def\PYZrb{]}
\makeatother


    % For linebreaks inside Verbatim environment from package fancyvrb. 
    \makeatletter
        \newbox\Wrappedcontinuationbox 
        \newbox\Wrappedvisiblespacebox 
        \newcommand*\Wrappedvisiblespace {\textcolor{red}{\textvisiblespace}} 
        \newcommand*\Wrappedcontinuationsymbol {\textcolor{red}{\llap{\tiny$\m@th\hookrightarrow$}}} 
        \newcommand*\Wrappedcontinuationindent {3ex } 
        \newcommand*\Wrappedafterbreak {\kern\Wrappedcontinuationindent\copy\Wrappedcontinuationbox} 
        % Take advantage of the already applied Pygments mark-up to insert 
        % potential linebreaks for TeX processing. 
        %        {, <, #, %, $, ' and ": go to next line. 
        %        _, }, ^, &, >, - and ~: stay at end of broken line. 
        % Use of \textquotesingle for straight quote. 
        \newcommand*\Wrappedbreaksatspecials {% 
            \def\PYGZus{\discretionary{\char`\_}{\Wrappedafterbreak}{\char`\_}}% 
            \def\PYGZob{\discretionary{}{\Wrappedafterbreak\char`\{}{\char`\{}}% 
            \def\PYGZcb{\discretionary{\char`\}}{\Wrappedafterbreak}{\char`\}}}% 
            \def\PYGZca{\discretionary{\char`\^}{\Wrappedafterbreak}{\char`\^}}% 
            \def\PYGZam{\discretionary{\char`\&}{\Wrappedafterbreak}{\char`\&}}% 
            \def\PYGZlt{\discretionary{}{\Wrappedafterbreak\char`\<}{\char`\<}}% 
            \def\PYGZgt{\discretionary{\char`\>}{\Wrappedafterbreak}{\char`\>}}% 
            \def\PYGZsh{\discretionary{}{\Wrappedafterbreak\char`\#}{\char`\#}}% 
            \def\PYGZpc{\discretionary{}{\Wrappedafterbreak\char`\%}{\char`\%}}% 
            \def\PYGZdl{\discretionary{}{\Wrappedafterbreak\char`\$}{\char`\$}}% 
            \def\PYGZhy{\discretionary{\char`\-}{\Wrappedafterbreak}{\char`\-}}% 
            \def\PYGZsq{\discretionary{}{\Wrappedafterbreak\textquotesingle}{\textquotesingle}}% 
            \def\PYGZdq{\discretionary{}{\Wrappedafterbreak\char`\"}{\char`\"}}% 
            \def\PYGZti{\discretionary{\char`\~}{\Wrappedafterbreak}{\char`\~}}% 
        } 
        % Some characters . , ; ? ! / are not pygmentized. 
        % This macro makes them "active" and they will insert potential linebreaks 
        \newcommand*\Wrappedbreaksatpunct {% 
            \lccode`\~`\.\lowercase{\def~}{\discretionary{\hbox{\char`\.}}{\Wrappedafterbreak}{\hbox{\char`\.}}}% 
            \lccode`\~`\,\lowercase{\def~}{\discretionary{\hbox{\char`\,}}{\Wrappedafterbreak}{\hbox{\char`\,}}}% 
            \lccode`\~`\;\lowercase{\def~}{\discretionary{\hbox{\char`\;}}{\Wrappedafterbreak}{\hbox{\char`\;}}}% 
            \lccode`\~`\:\lowercase{\def~}{\discretionary{\hbox{\char`\:}}{\Wrappedafterbreak}{\hbox{\char`\:}}}% 
            \lccode`\~`\?\lowercase{\def~}{\discretionary{\hbox{\char`\?}}{\Wrappedafterbreak}{\hbox{\char`\?}}}% 
            \lccode`\~`\!\lowercase{\def~}{\discretionary{\hbox{\char`\!}}{\Wrappedafterbreak}{\hbox{\char`\!}}}% 
            \lccode`\~`\/\lowercase{\def~}{\discretionary{\hbox{\char`\/}}{\Wrappedafterbreak}{\hbox{\char`\/}}}% 
            \catcode`\.\active
            \catcode`\,\active 
            \catcode`\;\active
            \catcode`\:\active
            \catcode`\?\active
            \catcode`\!\active
            \catcode`\/\active 
            \lccode`\~`\~ 	
        }
    \makeatother

    \let\OriginalVerbatim=\Verbatim
    \makeatletter
    \renewcommand{\Verbatim}[1][1]{%
        %\parskip\z@skip
        \sbox\Wrappedcontinuationbox {\Wrappedcontinuationsymbol}%
        \sbox\Wrappedvisiblespacebox {\FV@SetupFont\Wrappedvisiblespace}%
        \def\FancyVerbFormatLine ##1{\hsize\linewidth
            \vtop{\raggedright\hyphenpenalty\z@\exhyphenpenalty\z@
                \doublehyphendemerits\z@\finalhyphendemerits\z@
                \strut ##1\strut}%
        }%
        % If the linebreak is at a space, the latter will be displayed as visible
        % space at end of first line, and a continuation symbol starts next line.
        % Stretch/shrink are however usually zero for typewriter font.
        \def\FV@Space {%
            \nobreak\hskip\z@ plus\fontdimen3\font minus\fontdimen4\font
            \discretionary{\copy\Wrappedvisiblespacebox}{\Wrappedafterbreak}
            {\kern\fontdimen2\font}%
        }%
        
        % Allow breaks at special characters using \PYG... macros.
        \Wrappedbreaksatspecials
        % Breaks at punctuation characters . , ; ? ! and / need catcode=\active 	
        \OriginalVerbatim[#1,codes*=\Wrappedbreaksatpunct]%
    }
    \makeatother

    % Exact colors from NB
    \definecolor{incolor}{HTML}{303F9F}
    \definecolor{outcolor}{HTML}{D84315}
    \definecolor{cellborder}{HTML}{CFCFCF}
    \definecolor{cellbackground}{HTML}{F7F7F7}
    
    % prompt
    \makeatletter
    \newcommand{\boxspacing}{\kern\kvtcb@left@rule\kern\kvtcb@boxsep}
    \makeatother
    \newcommand{\prompt}[4]{
        {\ttfamily\llap{{\color{#2}[#3]:\hspace{3pt}#4}}\vspace{-\baselineskip}}
    }
    

    
    % Prevent overflowing lines due to hard-to-break entities
    \sloppy 
    % Setup hyperref package
    \hypersetup{
      breaklinks=true,  % so long urls are correctly broken across lines
      colorlinks=true,
      urlcolor=urlcolor,
      linkcolor=linkcolor,
      citecolor=citecolor,
      }
    % Slightly bigger margins than the latex defaults
    
    \geometry{verbose,tmargin=1in,bmargin=1in,lmargin=1in,rmargin=1in}
    
\begin{document}
    
\maketitle
    
\tableofcontents

\hypertarget{overview}{%
\section{Overview}\label{overview}}

We create a computational framework to implement Newton's method for
finding roots of polynomials. For the most basic, one variable, case
this looks like

\begin{equation}
x_{n + 1} = x_n - \frac{f(x_n)}{f'(x_n)}. 
\end{equation}

This algorithm finds the roots of a one-dimensional function. We want to
expand this method to solve \emph{systems} of equations. This is done by

\begin{equation}
\mathbf{x}_{n+1} = \mathbf{x_n} - J_F(\mathbf{x}_n)^{-1} F(\mathbf{x}_n)
\end{equation}

or, for sake of computation, we simply solve

\begin{equation}
J_F(\mathbf{x_n})(\mathbf{x}_{n+1} - \mathbf{x}_n) = -F(\mathbf{x}_n)
\end{equation}

for the quantity \(\mathbf{x}_{n+1} - \mathbf{x}_n\) so that we do not
have to compute the inverse of the Jacobian. We will write a function to
carry out the algorithm outlined above and will explore several examples
using the function we wrote. All computation will be done in
\texttt{Julia} with the packages found below

    \begin{tcolorbox}[breakable, size=fbox, boxrule=1pt, pad at break*=1mm,colback=cellbackground, colframe=cellborder]
\prompt{In}{incolor}{1}{\boxspacing}
\begin{Verbatim}[commandchars=\\\{\}]
\PY{c}{\PYZsh{} load packages}
\PY{k}{using} \PY{n}{HomotopyContinuation}
\end{Verbatim}
\end{tcolorbox}

    \begin{tcolorbox}[breakable, size=fbox, boxrule=1pt, pad at break*=1mm,colback=cellbackground, colframe=cellborder]
\prompt{In}{incolor}{2}{\boxspacing}
\begin{Verbatim}[commandchars=\\\{\}]
\PY{c}{\PYZsh{} declare variables x and y}
\PY{n+nd}{@var} \PY{n}{x} \PY{n}{y}
\PY{c}{\PYZsh{} define the polynomials}
\PY{n}{f₁} \PY{o}{=} \PY{p}{(}\PY{n}{x}\PY{o}{\PYZca{}}\PY{l+m+mi}{4} \PY{o}{+} \PY{n}{y}\PY{o}{\PYZca{}}\PY{l+m+mi}{4} \PY{o}{\PYZhy{}} \PY{l+m+mi}{1}\PY{p}{)} \PY{o}{*} \PY{p}{(}\PY{n}{x}\PY{o}{\PYZca{}}\PY{l+m+mi}{2} \PY{o}{+} \PY{n}{y}\PY{o}{\PYZca{}}\PY{l+m+mi}{2} \PY{o}{\PYZhy{}} \PY{l+m+mi}{2}\PY{p}{)} \PY{o}{+} \PY{n}{x}\PY{o}{\PYZca{}}\PY{l+m+mi}{5} \PY{o}{*} \PY{n}{y}
\PY{n}{f₂} \PY{o}{=} \PY{n}{x}\PY{o}{\PYZca{}}\PY{l+m+mi}{2} \PY{o}{+} \PY{l+m+mi}{2}\PY{o}{*}\PY{n}{x}\PY{o}{*}\PY{n}{y}\PY{o}{\PYZca{}}\PY{l+m+mi}{2} \PY{o}{\PYZhy{}} \PY{l+m+mi}{2}\PY{o}{*}\PY{n}{y}\PY{o}{\PYZca{}}\PY{l+m+mi}{2} \PY{o}{\PYZhy{}} \PY{l+m+mi}{1}\PY{o}{/}\PY{l+m+mi}{2}
\PY{n}{F} \PY{o}{=} \PY{n}{System}\PY{p}{(}\PY{p}{[}\PY{n}{f₁}\PY{p}{,} \PY{n}{f₂}\PY{p}{]}\PY{p}{)}
\PY{n}{result} \PY{o}{=} \PY{n}{solve}\PY{p}{(}\PY{n}{F}\PY{p}{)}
\end{Verbatim}
\end{tcolorbox}

    \begin{Verbatim}[commandchars=\\\{\}]
\textcolor{ansi-green}{Tracking 18 paths{\ldots} 100\%|██████████████████████████████| Time:
0:00:26}
\textcolor{ansi-blue}{  \# paths tracked:                  18}
\textcolor{ansi-blue}{  \# non-singular solutions (real):  18 (4)}
\textcolor{ansi-blue}{  \# singular endpoints (real):      0 (0)}
\textcolor{ansi-blue}{  \# total solutions (real):         18 (4)}
    \end{Verbatim}

            \begin{tcolorbox}[breakable, size=fbox, boxrule=.5pt, pad at break*=1mm, opacityfill=0]
\prompt{Out}{outcolor}{2}{\boxspacing}
\begin{Verbatim}[commandchars=\\\{\}]
Result with 18 solutions
========================
• 18 paths tracked
• 18 non-singular solutions (4 real)
• random\_seed: 0x6a9bef16
• start\_system: :polyhedral

\end{Verbatim}
\end{tcolorbox}

    \hypertarget{solving-systems-of-equations-using-newton-method}{%
\section{Solving Systems of Equations Using Newton
Method}\label{solving-systems-of-equations-using-newton-method}}

We now create a function to solve a system of equations using the Newton
method and the same principles as above. That is, given a system of
equations we find a solution by using the recursive formula (3) and solve for \(\mathbf{x}_{n+1} - \mathbf{x}_n\). We then extract the value for \(\mathbf{x}_{n + 1}\) and repeat the process until a
particular tolerance is reached. This process is carried out in the
following function:

   \begin{tcolorbox}[breakable, size=fbox, boxrule=1pt, pad at break*=1mm,colback=cellbackground, colframe=cellborder]
\prompt{In}{incolor}{3}{\boxspacing}
\begin{Verbatim}[commandchars=\\\{\}]
\PY{n}{solutions}\PY{p}{(}\PY{n}{result}\PY{p}{)}
\end{Verbatim}
\end{tcolorbox}

    
    \begin{Verbatim}[commandchars=\\\{\}]
18-element Vector\{Vector\{ComplexF64\}\}:
 [0.9443571312488813 + 0.3118635017972661im, 0.32083818529039954 + 0.9677296009728283im]
 [0.8999179208471728 - 8.474091755303838e-33im, -1.2441827613422727 + 9.244463733058732e-33im]
 [1.0866676911062136 - 0.3290700669978769im, -0.24048106708661118 + 1.1350215823993672im]
 [0.06944255588971958 - 1.0734210145259255im, 0.2852544211796043 + 0.7076856100161802im]
 [0.9443571312488815 - 0.311863501797266im, 0.3208381852903996 - 0.9677296009728285im]
 [0.8708494909007279 + 0.03055030123533535im, 0.9716347867303131 + 0.21528324968394982im]
 [-1.0665536440076708 + 0.1424944286215267im, -0.3998429331325364 + 0.07871238147493467im]
 [-1.671421392838003 + 0.0im, 0.6552051858720408 + 7.52316384526264e-37im]
 [1.7132941375582666 - 0.5813863945698412im, 0.047514134780854124 + 1.252792951007688im]
 [1.0866676911062132 + 0.3290700669978767im, -0.24048106708661127 - 1.1350215823993668im]
 [0.8708494909007278 - 0.03055030123533537im, 0.9716347867303131 - 0.2152832496839498im]
 [0.8209788924342627 - 1.88079096131566e-36im, -0.6971326459489459 - 6.018531076210112e-36im]
 [0.06944255588971956 + 1.0734210145259255im, 0.28525442117960437 - 0.7076856100161802im]
 [0.07565391048031057 + 0.9487419814734106im, -0.24800445792173503 + 0.6838307098593375im]
 [-0.9368979667963299 + 3.5264830524668625e-38im, 0.312284081738601 + 2.82118644197349e-37im]
 [-1.0665536440076708 - 0.1424944286215266im, -0.3998429331325363 - 0.0787123814749346im]
 [0.07565391048031057 - 0.9487419814734106im, -0.24800445792173503 - 0.6838307098593376im]
 [1.7132941375582666 + 0.5813863945698413im, 0.04751413478085409 - 1.252792951007688im]
    \end{Verbatim}
    
\subsection{A first example}

    We then verify our solution by considering the case of minimizing the
function

\begin{equation}
    f : (x, y) \mapsto 2x^4 + y^4 + 2xy^2 - x^2 - \frac{1}{3}y^2 + 10. 
\end{equation}

In order to do this, we take the gradient of the above and then use
\texttt{newton\_solve} to find the minimum. We note that the gradient is
given by

\begin{equation}
    \nabla (x, y) = (8x^3 + 2y^2 - 2x, 4y^3 + 4xy - \frac{2}{3}y). 
\end{equation}

We start from the point $p = (2, 2)$ and commence the algorithm.

    \begin{tcolorbox}[breakable, size=fbox, boxrule=1pt, pad at break*=1mm,colback=cellbackground, colframe=cellborder]
\prompt{In}{incolor}{3}{\boxspacing}
\begin{Verbatim}[commandchars=\\\{\}]
\PY{n}{f}\PY{p}{(}\PY{n}{x}\PY{p}{,} \PY{n}{y}\PY{p}{)} \PY{o}{=} \PY{p}{[}\PY{l+m+mi}{8}\PY{o}{*}\PY{n}{x}\PY{o}{\PYZca{}}\PY{l+m+mi}{3} \PY{o}{+} \PY{l+m+mi}{2}\PY{o}{*}\PY{n}{y}\PY{o}{\PYZca{}}\PY{l+m+mi}{2} \PY{o}{\PYZhy{}} \PY{l+m+mi}{2}\PY{o}{*}\PY{n}{x}\PY{p}{,} \PY{l+m+mi}{4}\PY{o}{*}\PY{n}{y}\PY{o}{\PYZca{}}\PY{l+m+mi}{3} \PY{o}{+} \PY{l+m+mi}{4}\PY{o}{*}\PY{n}{x}\PY{o}{*}\PY{n}{y} \PY{o}{\PYZhy{}} \PY{p}{(}\PY{l+m+mi}{2}\PY{o}{/}\PY{l+m+mi}{3}\PY{p}{)}\PY{o}{*}\PY{n}{y}\PY{p}{]}
\PY{n}{p} \PY{o}{=} \PY{p}{[}\PY{l+m+mf}{2.0}\PY{p}{,} \PY{l+m+mf}{2.0}\PY{p}{]}
\PY{n}{p\PYZus{}sol}\PY{p}{,} \PY{n}{steps\PYZus{}x}\PY{p}{,} \PY{n}{steps\PYZus{}y} \PY{o}{=} \PY{n}{newton\PYZus{}solve}\PY{p}{(}\PY{n}{p}\PY{p}{,} \PY{n}{f}\PY{p}{)}
\PY{n}{p\PYZus{}sol}
\end{Verbatim}
\end{tcolorbox}

            \begin{tcolorbox}[breakable, size=fbox, boxrule=.5pt, pad at break*=1mm, opacityfill=0]
\prompt{Out}{outcolor}{3}{\boxspacing}
\begin{Verbatim}[commandchars=\\\{\}]
2-element Vector\{Float64\}:
 0.5
 1.9510543672165753e-19
\end{Verbatim}
\end{tcolorbox}
        
    To show this is indeed a solution, we plug in our above solution and see
that we get zero.

    \begin{tcolorbox}[breakable, size=fbox, boxrule=1pt, pad at break*=1mm,colback=cellbackground, colframe=cellborder]
\prompt{In}{incolor}{4}{\boxspacing}
\begin{Verbatim}[commandchars=\\\{\}]
\PY{n}{f}\PY{p}{(}\PY{n}{p\PYZus{}sol}\PY{o}{...}\PY{p}{)}
\end{Verbatim}
\end{tcolorbox}

            \begin{tcolorbox}[breakable, size=fbox, boxrule=.5pt, pad at break*=1mm, opacityfill=0]
\prompt{Out}{outcolor}{4}{\boxspacing}
\begin{Verbatim}[commandchars=\\\{\}]
2-element Vector\{Float64\}:
 0.0
 2.601405822955434e-19
\end{Verbatim}
\end{tcolorbox}

We then plot to visualize what our algorithm is doing. 
        
    \begin{tcolorbox}[breakable, size=fbox, boxrule=1pt, pad at break*=1mm,colback=cellbackground, colframe=cellborder]
\prompt{In}{incolor}{5}{\boxspacing}
\begin{Verbatim}[commandchars=\\\{\}]
\PY{n}{f\PYZus{}surf1}\PY{p}{(}\PY{n}{x}\PY{p}{,} \PY{n}{y}\PY{p}{)} \PY{o}{=} \PY{l+m+mi}{8}\PY{o}{*}\PY{n}{x}\PY{o}{\PYZca{}}\PY{l+m+mi}{3} \PY{o}{+} \PY{l+m+mi}{2}\PY{o}{*}\PY{n}{y}\PY{o}{\PYZca{}}\PY{l+m+mi}{2} \PY{o}{\PYZhy{}} \PY{l+m+mi}{2}\PY{o}{*}\PY{n}{x}
\PY{n}{f\PYZus{}surf2}\PY{p}{(}\PY{n}{x}\PY{p}{,} \PY{n}{y}\PY{p}{)} \PY{o}{=} \PY{l+m+mi}{4}\PY{o}{*}\PY{n}{y}\PY{o}{\PYZca{}}\PY{l+m+mi}{3} \PY{o}{+} \PY{l+m+mi}{4}\PY{o}{*}\PY{n}{x}\PY{o}{*}\PY{n}{y} \PY{o}{\PYZhy{}} \PY{p}{(}\PY{l+m+mi}{2}\PY{o}{/}\PY{l+m+mi}{3}\PY{p}{)}\PY{o}{*}\PY{n}{y}
\PY{n}{scatter}\PY{p}{(}\PY{n}{steps\PYZus{}x}\PY{p}{,} \PY{n}{steps\PYZus{}y}\PY{p}{,} \PY{n}{color} \PY{o}{=} \PY{l+s}{\PYZdq{}}\PY{l+s}{red}\PY{l+s}{\PYZdq{}}\PY{p}{,} \PY{n}{label} \PY{o}{=} \PY{l+s}{\PYZdq{}}\PY{l+s}{Newton Algorithm}\PY{l+s}{\PYZdq{}}\PY{p}{,} \PY{n}{xlabel} \PY{o}{=} \PY{l+s+sa}{L}\PY{l+s}{\PYZdq{}}\PY{l+s}{x}\PY{l+s}{\PYZdq{}}\PY{p}{,} \PY{n}{ylabel} \PY{o}{=} \PY{l+s+sa}{L}\PY{l+s}{\PYZdq{}}\PY{l+s}{y}\PY{l+s}{\PYZdq{}}\PY{p}{,} \PY{n}{legend}\PY{o}{=}\PY{l+s+ss}{:bottomright}\PY{p}{)}
\PY{n}{implicit\PYZus{}plot!}\PY{p}{(}\PY{n}{f\PYZus{}surf1}\PY{p}{;} \PY{n}{xlims}\PY{o}{=}\PY{p}{(}\PY{o}{\PYZhy{}}\PY{l+m+mi}{2}\PY{p}{,}\PY{l+m+mi}{2}\PY{p}{)}\PY{p}{,} \PY{n}{ylims}\PY{o}{=}\PY{p}{(}\PY{o}{\PYZhy{}}\PY{l+m+mi}{2}\PY{p}{,}\PY{l+m+mi}{2}\PY{p}{)}\PY{p}{,} \PY{n}{label} \PY{o}{=} \PY{l+s+sa}{L}\PY{l+s}{\PYZdq{}}\PY{l+s}{\PYZbs{}}\PY{l+s}{partial f / }\PY{l+s}{\PYZbs{}}\PY{l+s}{partial x}\PY{l+s}{\PYZdq{}}\PY{p}{,} \PY{n}{line} \PY{o}{=} \PY{p}{(}\PY{l+s+ss}{:purple}\PY{p}{)}\PY{p}{)}
\PY{n}{implicit\PYZus{}plot!}\PY{p}{(}\PY{n}{f\PYZus{}surf2}\PY{p}{;} \PY{n}{xlims}\PY{o}{=}\PY{p}{(}\PY{o}{\PYZhy{}}\PY{l+m+mi}{2}\PY{p}{,}\PY{l+m+mi}{2}\PY{p}{)}\PY{p}{,} \PY{n}{ylims}\PY{o}{=}\PY{p}{(}\PY{o}{\PYZhy{}}\PY{l+m+mi}{2}\PY{p}{,}\PY{l+m+mi}{2}\PY{p}{)}\PY{p}{,} \PY{n}{label} \PY{o}{=} \PY{l+s+sa}{L}\PY{l+s}{\PYZdq{}}\PY{l+s}{\PYZbs{}}\PY{l+s}{partial f / }\PY{l+s}{\PYZbs{}}\PY{l+s}{partial y}\PY{l+s}{\PYZdq{}}\PY{p}{)}
\end{Verbatim}
\end{tcolorbox}
 
            
\prompt{Out}{outcolor}{5}{}
    
    \begin{center}
    \adjustimage{max size={0.9\linewidth}{0.9\paperheight}}{project_files/project_8_0.pdf}
    \end{center}
    { \hspace*{\fill} \\}
    
and we see that everything works well. 

    \hypertarget{finding-the-closest-point-to-the-statistical-independence-model-using-newton-method}{%
\subsection{Finding the closest point to the statistical independence
model using Newton
Method}\label{finding-the-closest-point-to-the-statistical-independence-model-using-newton-method}}

We implement our above algorithm to minimize the distance to the statistical
independence model with defining equations

\begin{equation}
\begin{split}
x_1 + x_2 + x_3 + x_4 & = 1 \\
x_1 x_4 - x_2 x_3 = 0
\end{split}
\end{equation}

To find the closest points to the curve defined by these equations, we
minimize the distance formula using Lagrange multipliers, that is, we
solve the system of equations given by

\[
\begin{split}
x_1 + x_2 + x_3 + x_4 & = 1 \\
x_1 x_4 - x_2 x_3 & = 0 \\
2(x_1 - 1) & = \lambda_1 + \lambda_2 x_4 \\
2(x_2 - 5) & = \lambda_1 - \lambda_2 x_3 \\
2(x_3 - 2) & = \lambda_1 - \lambda_2 x_2 \\
2(x_4 - 3) & = \lambda_1 + \lambda_2 x_1
\end{split}
\]

where we have considered the distance to the point \(p = (1, 5, 2, 3) \in \mathbb{R}^4\). This
can be done easily using \texttt{newton\_solve} as follows

    \begin{tcolorbox}[breakable, size=fbox, boxrule=1pt, pad at break*=1mm,colback=cellbackground, colframe=cellborder]
\prompt{In}{incolor}{6}{\boxspacing}
\begin{Verbatim}[commandchars=\\\{\}]
\PY{n}{f}\PY{p}{(}\PY{n}{x1}\PY{p}{,} \PY{n}{x2}\PY{p}{,} \PY{n}{x3}\PY{p}{,} \PY{n}{x4}\PY{p}{,} \PY{n}{λ1}\PY{p}{,} \PY{n}{λ2}\PY{p}{)} \PY{o}{=} 
    \PY{p}{[}\PY{n}{x1} \PY{o}{+} \PY{n}{x2} \PY{o}{+} \PY{n}{x3} \PY{o}{+} \PY{n}{x4} \PY{o}{\PYZhy{}} \PY{l+m+mi}{1}\PY{p}{,} 
    \PY{n}{x1}\PY{o}{*}\PY{n}{x4} \PY{o}{\PYZhy{}} \PY{n}{x2}\PY{o}{*}\PY{n}{x3}\PY{p}{,} 
    \PY{l+m+mi}{2}\PY{o}{*}\PY{p}{(}\PY{n}{x1} \PY{o}{\PYZhy{}} \PY{l+m+mi}{1}\PY{p}{)} \PY{o}{\PYZhy{}} \PY{n}{λ1} \PY{o}{\PYZhy{}} \PY{n}{λ2}\PY{o}{*}\PY{n}{x4}\PY{p}{,} 
    \PY{l+m+mi}{2}\PY{o}{*}\PY{p}{(}\PY{n}{x2} \PY{o}{\PYZhy{}} \PY{l+m+mi}{5}\PY{p}{)} \PY{o}{\PYZhy{}} \PY{n}{λ1} \PY{o}{+} \PY{n}{λ2}\PY{o}{*}\PY{n}{x3}\PY{p}{,} 
    \PY{l+m+mi}{2}\PY{o}{*}\PY{p}{(}\PY{n}{x3} \PY{o}{\PYZhy{}} \PY{l+m+mi}{2}\PY{p}{)} \PY{o}{\PYZhy{}} \PY{n}{λ1} \PY{o}{+} \PY{n}{λ2}\PY{o}{*}\PY{n}{x2}\PY{p}{,} 
    \PY{l+m+mi}{2}\PY{o}{*}\PY{p}{(}\PY{n}{x4} \PY{o}{\PYZhy{}} \PY{l+m+mi}{3}\PY{p}{)} \PY{o}{\PYZhy{}} \PY{n}{λ1} \PY{o}{\PYZhy{}} \PY{n}{λ2}\PY{o}{*}\PY{n}{x1}\PY{p}{]}
\PY{n}{p} \PY{o}{=} \PY{n}{randn}\PY{p}{(}\PY{l+m+mi}{6}\PY{p}{)}
\PY{n}{p\PYZus{}sol}\PY{p}{,} \PY{n}{steps\PYZus{}x}\PY{p}{,} \PY{n}{steps\PYZus{}y} \PY{o}{=} \PY{n}{newton\PYZus{}solve}\PY{p}{(}\PY{n}{p}\PY{p}{,} \PY{n}{f}\PY{p}{)}
\PY{n}{p\PYZus{}sol}
\end{Verbatim}
\end{tcolorbox}

            \begin{tcolorbox}[breakable, size=fbox, boxrule=.5pt, pad at break*=1mm, opacityfill=0]
\prompt{Out}{outcolor}{6}{\boxspacing}
\begin{Verbatim}[commandchars=\\\{\}]
6-element Vector\{Float64\}:
 -1.6019431755432922
  2.412543396773429
 -0.3743000249732863
  0.5636998037431494
 -5.117655226373716
 -0.1529734872005719
\end{Verbatim}
\end{tcolorbox}
        
    To verify this is a solution, we plug into the above system and find
that we get zero

    \begin{tcolorbox}[breakable, size=fbox, boxrule=1pt, pad at break*=1mm,colback=cellbackground, colframe=cellborder]
\prompt{In}{incolor}{7}{\boxspacing}
\begin{Verbatim}[commandchars=\\\{\}]
\PY{n}{f}\PY{p}{(}\PY{n}{p\PYZus{}sol}\PY{o}{...}\PY{p}{)}
\end{Verbatim}
\end{tcolorbox}

            \begin{tcolorbox}[breakable, size=fbox, boxrule=.5pt, pad at break*=1mm, opacityfill=0]
\prompt{Out}{outcolor}{7}{\boxspacing}
\begin{Verbatim}[commandchars=\\\{\}]
6-element Vector\{Float64\}:
  0.0
  0.0
  2.498001805406602e-16
  1.3877787807814457e-16
 -5.551115123125783e-17
  9.159339953157541e-16
\end{Verbatim}
\end{tcolorbox}

and we see that the point $p_{min} \approx (-1.6, 2.4, -0.37, 0.56)$ is the closest point on the surface to the point $p = (1,5,2,3)$. 

\break
        
    \hypertarget{finding-a-point-on-a-curve-using-newton-method}{%
\subsection{Finding a point on a curve using Newton
Method}\label{finding-a-point-on-a-curve-using-newton-method}}

We now find a point on the following curve using Newton

\[
(x, y) \mapsto (x^4 + y^4 - 1)(x^2 + y^2 - 2) + x^5y. 
\]

It turns out that simply starting from a point and applying
\texttt{newton\_solve} works in finding a point on the surface even in
the under-determined case. This is a result of the
\texttt{\textbackslash{}} operator automatically finding the norm of the
under-determined system created by taking the Jacobian of our function,
i.e.~\texttt{\textbackslash{}} automatically solves the system

\begin{equation}
Jx = f(p)
\end{equation}

for the minimum \(x\). This process is shown below

    \begin{tcolorbox}[breakable, size=fbox, boxrule=1pt, pad at break*=1mm,colback=cellbackground, colframe=cellborder]
\prompt{In}{incolor}{8}{\boxspacing}
\begin{Verbatim}[commandchars=\\\{\}]
\PY{n}{f}\PY{p}{(}\PY{n}{x}\PY{p}{,} \PY{n}{y}\PY{p}{)} \PY{o}{=} \PY{p}{[}\PY{p}{(}\PY{n}{x}\PY{o}{\PYZca{}}\PY{l+m+mi}{4} \PY{o}{+} \PY{n}{y}\PY{o}{\PYZca{}}\PY{l+m+mi}{4} \PY{o}{\PYZhy{}} \PY{l+m+mi}{1}\PY{p}{)} \PY{o}{*} \PY{p}{(}\PY{n}{x}\PY{o}{\PYZca{}}\PY{l+m+mi}{2} \PY{o}{+} \PY{n}{y}\PY{o}{\PYZca{}}\PY{l+m+mi}{2} \PY{o}{\PYZhy{}} \PY{l+m+mi}{2}\PY{p}{)} \PY{o}{+} \PY{p}{(}\PY{n}{x}\PY{o}{\PYZca{}}\PY{l+m+mi}{5}\PY{p}{)} \PY{o}{*} \PY{n}{y}\PY{p}{]}
\PY{n}{p} \PY{o}{=} \PY{p}{[}\PY{l+m+mf}{6.0}\PY{p}{,} \PY{l+m+mf}{6.0}\PY{p}{]}
\PY{n}{p\PYZus{}sol}\PY{p}{,} \PY{n}{steps\PYZus{}x}\PY{p}{,} \PY{n}{steps\PYZus{}y} \PY{o}{=} \PY{n}{newton\PYZus{}solve}\PY{p}{(}\PY{n}{p}\PY{p}{,} \PY{n}{f}\PY{p}{)}
\PY{n}{p\PYZus{}sol}
\end{Verbatim}
\end{tcolorbox}

            \begin{tcolorbox}[breakable, size=fbox, boxrule=.5pt, pad at break*=1mm, opacityfill=0]
\prompt{Out}{outcolor}{8}{\boxspacing}
\begin{Verbatim}[commandchars=\\\{\}]
2-element Vector\{Float64\}:
 0.7143919259772923
 1.087335033742158
\end{Verbatim}
\end{tcolorbox}
        
    We then plot the steps taken by the algorithm and plot the surface to
visually see what our algorithm is doing.

    \begin{tcolorbox}[breakable, size=fbox, boxrule=1pt, pad at break*=1mm,colback=cellbackground, colframe=cellborder]
\prompt{In}{incolor}{9}{\boxspacing}
\begin{Verbatim}[commandchars=\\\{\}]
\PY{n}{f\PYZus{}surf3}\PY{p}{(}\PY{n}{x}\PY{p}{,} \PY{n}{y}\PY{p}{)} \PY{o}{=} \PY{p}{(}\PY{n}{x}\PY{o}{\PYZca{}}\PY{l+m+mi}{4} \PY{o}{+} \PY{n}{y}\PY{o}{\PYZca{}}\PY{l+m+mi}{4} \PY{o}{\PYZhy{}} \PY{l+m+mi}{1}\PY{p}{)} \PY{o}{*} \PY{p}{(}\PY{n}{x}\PY{o}{\PYZca{}}\PY{l+m+mi}{2} \PY{o}{+} \PY{n}{y}\PY{o}{\PYZca{}}\PY{l+m+mi}{2} \PY{o}{\PYZhy{}} \PY{l+m+mi}{2}\PY{p}{)} \PY{o}{+} \PY{p}{(}\PY{n}{x}\PY{o}{\PYZca{}}\PY{l+m+mi}{5}\PY{p}{)} \PY{o}{*} \PY{n}{y}
\PY{n}{scatter}\PY{p}{(}\PY{n}{steps\PYZus{}x}\PY{p}{,} \PY{n}{steps\PYZus{}y}\PY{p}{,} \PY{n}{color} \PY{o}{=} \PY{l+s}{\PYZdq{}}\PY{l+s}{red}\PY{l+s}{\PYZdq{}}\PY{p}{,} \PY{n}{label} \PY{o}{=} \PY{l+s}{\PYZdq{}}\PY{l+s}{Newton Algorithm}\PY{l+s}{\PYZdq{}}\PY{p}{,} \PY{n}{legend}\PY{o}{=}\PY{l+s+ss}{:bottomright}\PY{p}{,} 
            \PY{n}{xlabel} \PY{o}{=} \PY{l+s+sa}{L}\PY{l+s}{\PYZdq{}}\PY{l+s}{x}\PY{l+s}{\PYZdq{}}\PY{p}{,} \PY{n}{ylabel} \PY{o}{=} \PY{l+s+sa}{L}\PY{l+s}{\PYZdq{}}\PY{l+s}{y}\PY{l+s}{\PYZdq{}}\PY{p}{)}
\PY{n}{implicit\PYZus{}plot!}\PY{p}{(}\PY{n}{f\PYZus{}surf3}\PY{p}{;} \PY{n}{xlims}\PY{o}{=}\PY{p}{(}\PY{o}{\PYZhy{}}\PY{l+m+mi}{2}\PY{p}{,}\PY{l+m+mi}{6}\PY{p}{)}\PY{p}{,} \PY{n}{ylims}\PY{o}{=}\PY{p}{(}\PY{o}{\PYZhy{}}\PY{l+m+mi}{2}\PY{p}{,}\PY{l+m+mi}{6}\PY{p}{)}\PY{p}{,} \PY{n}{label} \PY{o}{=} \PY{l+s}{\PYZdq{}}\PY{l+s}{Surface}\PY{l+s}{\PYZdq{}}\PY{p}{)}
\end{Verbatim}
\end{tcolorbox}
 
            
\prompt{Out}{outcolor}{9}{}
    
    \begin{center}
    \adjustimage{max size={0.9\linewidth}{0.9\paperheight}}{project_files/project_16_0.pdf}
    \end{center}
    { \hspace*{\fill} \\}
    
    Finally, we make sure our point lies on the curve by

    \begin{tcolorbox}[breakable, size=fbox, boxrule=1pt, pad at break*=1mm,colback=cellbackground, colframe=cellborder]
\prompt{In}{incolor}{10}{\boxspacing}
\begin{Verbatim}[commandchars=\\\{\}]
\PY{n}{f}\PY{p}{(}\PY{n}{p\PYZus{}sol}\PY{o}{...}\PY{p}{)}
\end{Verbatim}
\end{tcolorbox}


            \begin{tcolorbox}[breakable, size=fbox, boxrule=.5pt, pad at break*=1mm, opacityfill=0]
\prompt{Out}{outcolor}{10}{\boxspacing}
\begin{Verbatim}[commandchars=\\\{\}]
1-element Vector\{Float64\}:
 6.106226635438361e-16
\end{Verbatim}
\end{tcolorbox}
        
    \hypertarget{cases-of-non-convergence}{%
\subsection{Cases of non-convergence}\label{cases-of-non-convergence}}

Consider a system of equations that does \emph{not} converge such as

\begin{equation}
f : (x, y) \mapsto (x^2 + y^2 - 1, xy - 5)
\end{equation}

This can be visualized with the following plot:

    \begin{tcolorbox}[breakable, size=fbox, boxrule=1pt, pad at break*=1mm,colback=cellbackground, colframe=cellborder]
\prompt{In}{incolor}{11}{\boxspacing}
\begin{Verbatim}[commandchars=\\\{\}]
\PY{n}{circle}\PY{p}{(}\PY{n}{x}\PY{p}{,} \PY{n}{y}\PY{p}{)} \PY{o}{=} \PY{n}{x}\PY{o}{\PYZca{}}\PY{l+m+mi}{2} \PY{o}{+} \PY{n}{y}\PY{o}{\PYZca{}}\PY{l+m+mi}{2} \PY{o}{\PYZhy{}} \PY{l+m+mi}{1}
\PY{n}{hyperbola}\PY{p}{(}\PY{n}{x}\PY{p}{,} \PY{n}{y}\PY{p}{)} \PY{o}{=} \PY{n}{x}\PY{o}{*}\PY{n}{y} \PY{o}{\PYZhy{}} \PY{l+m+mi}{5}
\PY{n}{implicit\PYZus{}plot}\PY{p}{(}\PY{n}{circle}\PY{p}{;} \PY{n}{xlims} \PY{o}{=} \PY{p}{(}\PY{o}{\PYZhy{}}\PY{l+m+mi}{5}\PY{p}{,} \PY{l+m+mi}{5}\PY{p}{)}\PY{p}{,} \PY{n}{ylims} \PY{o}{=} \PY{p}{(}\PY{o}{\PYZhy{}}\PY{l+m+mi}{5}\PY{p}{,} \PY{l+m+mi}{5}\PY{p}{)}\PY{p}{,} \PY{n}{label} \PY{o}{=} \PY{l+s+sa}{L}\PY{l+s}{\PYZdq{}}\PY{l+s}{x\PYZca{}2 + y\PYZca{}2 = 1}\PY{l+s}{\PYZdq{}}\PY{p}{,} \PY{n}{legend}\PY{o}{=}\PY{l+s+ss}{:bottomright}\PY{p}{,} \PY{n}{xlabel} \PY{o}{=} \PY{l+s+sa}{L}\PY{l+s}{\PYZdq{}}\PY{l+s}{x}\PY{l+s}{\PYZdq{}}\PY{p}{,} \PY{n}{ylabel} \PY{o}{=} \PY{l+s+sa}{L}\PY{l+s}{\PYZdq{}}\PY{l+s}{y}\PY{l+s}{\PYZdq{}}\PY{p}{)}
\PY{n}{implicit\PYZus{}plot!}\PY{p}{(}\PY{n}{hyperbola}\PY{p}{;} \PY{n}{xlims} \PY{o}{=} \PY{p}{(}\PY{o}{\PYZhy{}}\PY{l+m+mi}{5}\PY{p}{,} \PY{l+m+mi}{5}\PY{p}{)}\PY{p}{,} \PY{n}{ylims} \PY{o}{=} \PY{p}{(}\PY{o}{\PYZhy{}}\PY{l+m+mi}{5}\PY{p}{,} \PY{l+m+mi}{5}\PY{p}{)}\PY{p}{,} \PY{n}{label} \PY{o}{=} \PY{l+s+sa}{L}\PY{l+s}{\PYZdq{}}\PY{l+s}{xy \PYZhy{} 5}\PY{l+s}{\PYZdq{}}\PY{p}{,} \PY{n}{line} \PY{o}{=} \PY{p}{(}\PY{l+s+ss}{:purple}\PY{p}{)}\PY{p}{)}
\end{Verbatim}
\end{tcolorbox}
 
            
\prompt{Out}{outcolor}{11}{}
    
    \begin{center}
    \adjustimage{max size={0.9\linewidth}{0.9\paperheight}}{project_files/project_20_0.pdf}
    \end{center}
    { \hspace*{\fill} \\}
    
    If we apply \texttt{newton\_solve} to this system this clearly throws an
error as shown below

    \begin{tcolorbox}[breakable, size=fbox, boxrule=1pt, pad at break*=1mm,colback=cellbackground, colframe=cellborder]
\prompt{In}{incolor}{12}{\boxspacing}
\begin{Verbatim}[commandchars=\\\{\}]
\PY{n}{f}\PY{p}{(}\PY{n}{x}\PY{p}{,} \PY{n}{y}\PY{p}{)} \PY{o}{=} \PY{p}{[}\PY{n}{x} \PY{o}{*} \PY{n}{y} \PY{o}{\PYZhy{}} \PY{l+m+mi}{5}\PY{p}{,} \PY{n}{x}\PY{o}{\PYZca{}}\PY{l+m+mi}{2} \PY{o}{+} \PY{n}{y}\PY{o}{\PYZca{}}\PY{l+m+mi}{2} \PY{o}{\PYZhy{}} \PY{l+m+mi}{1}\PY{p}{]}
\PY{n}{p} \PY{o}{=} \PY{n}{randn}\PY{p}{(}\PY{l+m+mi}{2}\PY{p}{)}
\PY{n}{newton\PYZus{}solve}\PY{p}{(}\PY{n}{p}\PY{p}{,} \PY{n}{f}\PY{p}{)}
\end{Verbatim}
\end{tcolorbox}

    \begin{Verbatim}[commandchars=\\\{\}, frame=single, framerule=2mm, rulecolor=\color{outerrorbackground}]
Algorithm did not converge!

Stacktrace:
 [1] error(s::String)
   @ Base ./error.jl:33
 [2] newton\_solve(start::Vector\{Float64\}, Func::Function, tolerance::Float64)
   @ Main ./In[2]:20
 [3] newton\_solve(start::Vector\{Float64\}, Func::Function)
   @ Main ./In[2]:4
 [4] top-level scope
   @ In[12]:3
 [5] eval
   @ ./boot.jl:373 [inlined]
 [6] include\_string(mapexpr::typeof(REPL.softscope), mod::Module, code::String, filename::String)
   @ Base ./loading.jl:1196
    \end{Verbatim}

    Note that this error is something we wrote into our function when
convergence does not occur, i.e.~when the number of steps taken is
greater than 500. To fix this problem we could consider moving to the
complex numbers, however, \texttt{ForwardDiff} does not support complex
numbers, thus we would need to use a different function to find the
gradient.

For a visualization of what's going on, we plot the first ten points
computed by \texttt{newton\_solve} and display them below.

    \begin{tcolorbox}[breakable, size=fbox, boxrule=1pt, pad at break*=1mm,colback=cellbackground, colframe=cellborder]
\prompt{In}{incolor}{13}{\boxspacing}
\begin{Verbatim}[commandchars=\\\{\}]
\PY{n}{f}\PY{p}{(}\PY{n}{x}\PY{p}{,} \PY{n}{y}\PY{p}{)} \PY{o}{=} \PY{p}{[}\PY{n}{x} \PY{o}{*} \PY{n}{y} \PY{o}{\PYZhy{}} \PY{l+m+mi}{5}\PY{p}{,} \PY{n}{x}\PY{o}{\PYZca{}}\PY{l+m+mi}{2} \PY{o}{+} \PY{n}{y}\PY{o}{\PYZca{}}\PY{l+m+mi}{2} \PY{o}{\PYZhy{}} \PY{l+m+mi}{1}\PY{p}{]}
\PY{n}{p} \PY{o}{=} \PY{p}{[}\PY{o}{\PYZhy{}}\PY{l+m+mf}{1.0}\PY{p}{,} \PY{l+m+mf}{3.0}\PY{p}{]}
\PY{n}{p\PYZus{}sol}\PY{p}{,} \PY{n}{steps\PYZus{}x}\PY{p}{,} \PY{n}{steps\PYZus{}y} \PY{o}{=} \PY{n}{newton\PYZus{}solve}\PY{p}{(}\PY{n}{p}\PY{p}{,} \PY{n}{f}\PY{p}{)}
\PY{n}{p\PYZus{}sol}
\end{Verbatim}
\end{tcolorbox}

    
    \begin{Verbatim}[commandchars=\\\{\}]
2-element Vector\{Float64\}:
  5.25963298444844
 -1.94300819409304
    \end{Verbatim}
    \begin{tcolorbox}[breakable, size=fbox, boxrule=1pt, pad at break*=1mm,colback=cellbackground, colframe=cellborder]
\prompt{In}{incolor}{14}{\boxspacing}
\begin{Verbatim}[commandchars=\\\{\}]
\PY{n}{scatter}\PY{p}{(}\PY{n}{steps\PYZus{}x}\PY{p}{,} \PY{n}{steps\PYZus{}y}\PY{p}{,} \PY{n}{color} \PY{o}{=} \PY{l+s}{\PYZdq{}}\PY{l+s}{red}\PY{l+s}{\PYZdq{}}\PY{p}{,} \PY{n}{label} \PY{o}{=} \PY{l+s}{\PYZdq{}}\PY{l+s}{Newton Algorithm}\PY{l+s}{\PYZdq{}}\PY{p}{)}
\PY{n}{implicit\PYZus{}plot!}\PY{p}{(}\PY{n}{circle}\PY{p}{;} \PY{n}{xlims} \PY{o}{=} \PY{p}{(}\PY{o}{\PYZhy{}}\PY{l+m+mi}{5}\PY{p}{,} \PY{l+m+mi}{5}\PY{p}{)}\PY{p}{,} \PY{n}{ylims} \PY{o}{=} \PY{p}{(}\PY{o}{\PYZhy{}}\PY{l+m+mi}{5}\PY{p}{,} \PY{l+m+mi}{5}\PY{p}{)}\PY{p}{,} \PY{n}{label} \PY{o}{=} \PY{l+s+sa}{L}\PY{l+s}{\PYZdq{}}\PY{l+s}{x\PYZca{}2 + y\PYZca{}2 = 1}\PY{l+s}{\PYZdq{}}\PY{p}{,} \PY{n}{legend}\PY{o}{=}\PY{l+s+ss}{:bottomright}\PY{p}{,} \PY{n}{xlabel} \PY{o}{=} \PY{l+s+sa}{L}\PY{l+s}{\PYZdq{}}\PY{l+s}{x}\PY{l+s}{\PYZdq{}}\PY{p}{,} \PY{n}{ylabel} \PY{o}{=} \PY{l+s+sa}{L}\PY{l+s}{\PYZdq{}}\PY{l+s}{y}\PY{l+s}{\PYZdq{}}\PY{p}{)}
\PY{n}{implicit\PYZus{}plot!}\PY{p}{(}\PY{n}{hyperbola}\PY{p}{;} \PY{n}{xlims} \PY{o}{=} \PY{p}{(}\PY{o}{\PYZhy{}}\PY{l+m+mi}{5}\PY{p}{,} \PY{l+m+mi}{5}\PY{p}{)}\PY{p}{,} \PY{n}{ylims} \PY{o}{=} \PY{p}{(}\PY{o}{\PYZhy{}}\PY{l+m+mi}{5}\PY{p}{,} \PY{l+m+mi}{5}\PY{p}{)}\PY{p}{,} \PY{n}{label} \PY{o}{=} \PY{l+s+sa}{L}\PY{l+s}{\PYZdq{}}\PY{l+s}{xy \PYZhy{} 5}\PY{l+s}{\PYZdq{}}\PY{p}{,} \PY{n}{line} \PY{o}{=} \PY{p}{(}\PY{l+s+ss}{:purple}\PY{p}{)}\PY{p}{)}
\end{Verbatim}
\end{tcolorbox}

    \begin{center}
    \adjustimage{max size={0.9\linewidth}{0.9\paperheight}}{project_files/project_25_0.pdf}
    \end{center}
    { \hspace*{\fill} \\}
    
    From this we can clearly see that the algorithm will not converge and
will rather eventually step off into infinity.

\section{Conclusions}

The above examples illustrate how we can use the Newton-Ralphson method
to find solutions to systems of multivariate equations. This method
clearly has flaws. Consider our first example in section 1.1. We can see
from the picture created that there should be a total of \emph{six}
solutions. We could find all of these by changing our starting point,
but this becomes tedious and, for higher dimensional objects, may not be
possible. Clearly a better method is needed in finding solutions, or our
algorithm must be improved.

Despite these obvious problems, Newton's method is incredibly fast and
the algorithm, when convergence is possible, converges very quickly.
Other methods that are more powerful, such as
\texttt{Homotopy\ Continuation}, can take a very long time to run. Thus,
we conclude that while Newton is a very powerful tool, other methods are
needed to more fully explore systems of equations.
    
    
\end{document}
